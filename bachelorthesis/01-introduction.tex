% 1. Introduction

\chapter{Introduction}
\label{chap:intro}
\thispagestyle{empty}
\newpage

Ion temperature gradient driven turbulence close to marginal stability exhibits zonal flow pattern formation on mesoscales, so-called $\exb$ staircase structures \cite{pradalier2010}.
Such pattern formation has been observed in local gradient-driven flux-tube simulations \cite{peeters2016, weikl2017, rath2021} as well as global gradient-driven \cite{mcmillan2009, villard2013, seo2022} and global flux-driven \cite{pradalier2010, pradalier2015, wang2020, kim2022, kishimoto2023} studies. 
In global studies, spanning a larger fraction of the minor radius, multiple radial repetitions of staircase structures are usually observed, with a typical pattern size of several ten Larmor radii.
% By contrast, the radial domain size of local flux-tube simulations is often restricted to a similar mesoscale.
By contrast, in the aforementioned local studies the radial size of $\exb$ staircase structures is always found to converge to the radial box size of the flux tube domain.
The above observations lead to the question: 
\textit{Does the basic pattern size always converges to the box size, or is there a typical mesoscale size inherent to staircase structures also in a local flux-tube description?}
The latter case would imply that it is not necessarily global physics, i.e., profile effects, that set (i) the radial size of the $\exb$ staircase pattern and (ii) the scale of avalanche-like transport events. These transport events are usually restricted to $\exb$ staircase structures and considered as a nonlocal transport mechanism \cite{pradalier2010}. 
In this brief communication the above question is addressed through a box size convergence scan of the same cases close to the nonlinear threshold for turbulence generation as studied in Ref. \citenum{peeters2016}.