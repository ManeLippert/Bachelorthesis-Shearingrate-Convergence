\newpage
\section{Gyrokinetic Theory}
\label{sec:gyrokinetic}

\subsection{Vlasov Equation}
\label{sub:vlasov}

Because of the large number of particles in the fusion plasma a prediction on the basis of Newton-Maxwell dynamics results in an impossible task for simulation, but this problem can be solved with a statistical approach. For that the particle density distribution function $f(\vect{x}, \vect{v}, t)$ in the six dimensional phase space $\{\vect{x}$, $\vect{v}\}$ with the particles position $\vect{x}$ and velocity $\vect{v}$ is needed. Because collisions are happening at much smaller frequencies than the characteristic frequenies connected to turbulence, the collisionless model is often preferred \cite{Garbet2010} which results through evolution of the particle density distribution function in the \textit{Vlasov equation}
\begin{gather}
	\frac{\partial f}{\partial t} + \frac{\partial f}{\partial \vect{x}} \cdot \frac{\mathrm{d}\vect{x}}{\mathrm{d}t} + \frac{\partial f}{\partial \vect{v}} \cdot \frac{\mathrm{d}\vect{v}}{\mathrm{d}t} = 0~.
	\label{eq:vlasov}
\end{gather}
To obtain a closed system the Maxwell equations with the particle density $n$ and current density $j$ can be described with the distribution function as follows
\begin{gather}
	n = \int \mathrm{d}\vect{v}\,f(\vect{x}, \vect{v}, t) \qquad j = q \int \mathrm{d}\vect{v}\, v f(\vect{x}, \vect{v}, t)~,
	\label{eq:densitycurrent}
\end{gather}
which are then substituted into the Maxwell equations
\begin{gather}
	\begin{aligned}
		\nabla \cdot \vect{B} &= 0 &\qquad \nabla \times \vect{B} &= \mu_0\left( \sum_\mathrm{species} j + \epsilon_0 \frac{\partial \vect{E}}{\partial t} \right) \\
		\nabla \cdot \vect{E} &= \frac{1}{\epsilon_0} \sum_\mathrm{species} qn &\qquad \nabla \times \vect{E} &= - \frac{\partial \vect{B}}{\partial t}~.
	\end{aligned}
	\label{eq:maxwell}
\end{gather}
The Vlasov equation (\ref{eq:vlasov}) in combination with the Maxwell equations (\ref{eq:densitycurrent}) and (\ref{eq:maxwell}) is the basis of the gyrokinetic model. \cite{Krommes2012}

\newpage

\subsection{Gyrokinetic Ordering}
\label{sub:gyroordering}

The typical spatio-temporal scales connected to the dynamics in a tokamak plasma allow for the so-called \textit{gyrokinetic ordering} as outlined below. The fast gyromotion is much smaller compared to typical  time scales connected to turbulence ($\omega/\omega_\mathrm{c} \sim 10^{-3})$. The length scales of the turbulence are associated with the wave vector $\vect{k}$ which can be seperated into a perpendicular component $k_\perp = |\vect{k} \times \vect{b}|$ and a parallel component $k_\parallel = |\vect{k} \cdot \vect{b}|$ where $\vect{b}$ is parallel to the poloidal component of the magnetic field. The perpendicular component $k_\perp$ is of the ordre of the thermal Larmor radius $k_\perp^{-1} \sim \rhoth$ which is significantly smaller than the scale on which the equilibrium density $n_0$ varies, which can be expressed with the gradient length $L_\mathrm{n} = |\nabla \ln n_0 |^{-1}$. The gradient length compares with the machine size $R$ which leads to the normalized Larmor radius $\rho_* = \rhoth/R \sim 10^{-3} - 10^{-4}$ with $R$ as major radius of the tokamak. The parallel component $k_\parallel$ on the other hand scales with the machine size $R$. Experiments show that in the core plasma the fluctuation amplitude of the density perturbation $\delta n/n_0$ and magnetic field fluctuation $\delta B/B_0$, with $B_0$ as strength of the magnetic field in equilibrium, is of order $\sim 10^{-4}$. Together this seperation result in the following
\begin{gather}
	\frac{\omega}{\omega_\mathrm{c}} \sim \frac{k_\parallel}{k_\perp} \sim \frac{\rhoth}{L_\mathrm{n}} \sim \frac{\delta n}{n_0} \sim \frac{\delta B}{B_0} \sim \frac{v_\mathrm{d}}{\vth} \sim \epsilon_\mathrm{g}
	\label{eq:gyroordering}
\end{gather}
with $\epsilon_\mathrm{g} \ll 1$ which applies to the typical dynamics of a fusion core plasma. \cite{Brizard2007,Garbet2010} The gyrokinetic ordering allows tallows to formulate reduced governing equations referred to as the gyrokinetic formalism.

\newpage

\subsection{Gyrokinetic Equation}
\label{sub:gyrokinetic}
 
For the description of charged particle behaviour in the tokamak device the \textit{guidingcenter coordinates} are used [Fig. \ref{fig:gyrocenterCoords}]. In this set of coordinates the guidingcenter follows the magnetic field with the parallel velocity $v_{\parallel}$. The gyro motion is described together with the magnetic moment $\mu = \frac{1}{2}mv_\perp/\omega_\mathrm{c}$, the gyrocenter $\vect{X}$ and the gyro phase $\zeta$ which gives a parameter set of six quantities $\{\vect{X}, v_{\parallel}, \mu, \zeta\}$. The gyrocenter position $\vect{X}$ is expressed with the particle position $\vect{x}$ and the postion Larmor radius vector $\vect{\rho_\mathrm{L}}$ as $\vect{X} = \vect{x} - \vect{\rho_\mathrm{L}}$.

\includegraphicsHere{Theory/Gyro-Center-Coordinates.pdf}{
	Sketch of gyrocenter coordinates where the charged particle performs a circular motion around the gyrocenter. \cite{Krommes2000}
}{fig:gyrocenterCoords}{0.7}

In the next step the unperturbed Lagrangian is expressed through the guidingcenter coordinates. The particle velocity $\vect{v}$ is decomposed into a parallel component $v_\parallel = \vect{v} \cdot \vect{b}$ and a perpendicular component $v_\perp = |\vect{v} \times \vect{b}|$ which relates to the magnetic moment $\mu$. As final step the Lagrangian is transformed into gyrocenter phase space, based on the so-called Lie perturbation theory. This method allows to eliminate the gyrophase-dependent contributions, rendering the gyrophase $\zeta$ an ignorable coordinate and the magnetic moment $\mu$ an exact constant of motion. \cite{Garbet2010,Cary1981,Cary1983} 
Taking everything into account the new coordinates of the Vlasov equation are $\{\vect{X}, v_{\parallel}, \mu\}$ which transforms the Vlasov equation into the \textit{gyrokinetic equation}, describing the evolution of the gyrocenter distribution function.
\begin{gather}
	\frac{\partial f}{\partial t} + \frac{\partial f}{\partial \vect{X}} \cdot \frac{\mathrm{d}\vect{X}}{\mathrm{d}t} + \frac{\partial f}{\partial v_\parallel} \cdot \frac{\mathrm{d}v_\parallel}{\mathrm{d}t} = 0~.
	\label{eq:gyrokinetic}
\end{gather}
The time derivative of the magnetic moment $\mu$ is zero because the magnetic moment is an exact invariant. 

\subsection{$\delta f$ Approximation and Local Limit}
\label{sub:application}

All simulations in this thesis are performed with GKW (Gyro Kinetic Workshop) which solves the gyrokinetic equationin the $\delta f$ approximation and local limit. For the $\delta f$ approximation the particle density distribution function f is seperated into perturbation $\delta f$ and equilibrium $f_0$ (constant in time) as expressed in the gyrokinetic ordering (\ref{eq:gyroordering}). This is given by
\begin{gather}
	f = f_0 + \delta f \qquad \delta f \sim \rho_* f_0~.
\end{gather}
In GKW the equilibrium distribution function $f_0$ is set to the Maxwell distribution function. The perturbed distribution function $\delta f$ is considered to vary perpendicular to the poloidal magnetic field of order of the Larmor radius while the equilibrium changes with the system size which results in the ordering
\begin{gather}
	\nabla_\perp(\delta f) = \nabla_\perp (f_0)~.
\end{gather}
The $\delta f$ approximation applies for global and local description where the latter is used in this thesis.
The \textit{local limit} on the other hand uses the spatial seperation of equilibrium and perturbation and implies that all equilibrium and geometry quantities are considered constant over the radial extent of the simulation volume. Because of the tokamak symmetry all equilibrium quantities are also constant in the toridial direction where the dependency of the equilibrium magnetic field on the poloidal angle is retrained in general. \\\bigskip

In addition to this, in GKW the following coordinates get applied with the parameter $\vect{X} = \{\xcoord, \ycoord, s\}$. Here, $\xcoord$ is the radial coordinate that labels the flux surfaces normalized by the thermal Larmor radius $\rhoth$, $\ycoord$ labels the field lines and is an approximate binormal coordinate. Together with the coordinate $s$ which parameterizes the length along the field lines and is referred to as the parallel coordinate these quantities form the Hamada coordinates \cite{Hamada1958}. This results in the coordinates $\{\xcoord, \ycoord, s, v_\parallel, \mu\}$. Combining the $\delta f$ approximation and the Hamada coordinates the gyrokinetic equation can be formulated as follows
\begin{gather}
	\frac{\partial g}{\partial t} + \vect{v}_\chi \cdot \nabla g + (v_\parallel \vect{b} + \vect{v}_\mathrm{D}) \cdot \nabla(\delta f) - \frac{\mu B}{m}\frac{\vect{B}\cdot\nabla B}{B^2}\frac{\partial (\delta f)}{\partial v_\parallel} = S~,
\end{gather}
where $g$ is a function which contains the perturbation $\delta f$ and the equilibrium $f_0$. Furthermore, $\vect{v}_\mathrm{D}$ and $\vect{v}_\chi$ are drift velocities which are caused by the inhomogeneous magnetic field ($\vect{v}_\mathrm{D}$) and through the $\exb$ drift ($\vect{v}_\chi$). The source term $S$ contains the distribution function in the equilibrium $f_0$ and a correction term for the collisions as well as the energy injection term. Due to prediodic boundary conditions in the radial direction the radial averaged gradients are fixed in time which is also referred to as \textit{gradient driven} approach. \cite{Peeters2009}