% 4. Results

\newpage
\section{The finite Heat Flux Threshold}
\label{sec:threshold}

In the final test the inverse background temperature gradient length $\rlt$ is varied at fixed $3\times3$ box size.
Since suppression of turbulence usually occurs at later times when approaching the finite heat flux threshold from below \cite{Peeters2016}, the analysis aims to lengthen the phase during which the zonal flow varies in time due to turbulent Reynolds stresses.
This scan covers realizations with
\begin{gather*}
	\rlt \in [6.0,~6.2,~6.4]~.
\end{gather*}
% what is the finite heat flux threshold
In the case of $\rlt = 6.2$ turbulence suppression is observed for $t > 11000\,R/\vth$, while stationary turbulence during the entire simulation time trace of $12000\,R/\vth$ is found for $\rlt = 6.4$ [Fig. \ref{fig:eflux-6.0-6.2-6.4-comparison}].
The finite heat flux threshold, hence, is
\begin{gather*}
	\boxed{\rlt|_\mathrm{finite} = 6.3 \pm 0.1}
\end{gather*}
in accordance to Ref. \citenum{Peeters2016}.
% Which stationay radial size occurs????
Although the initial quasi-stationary turbulence in the former case is significantly longer compared to the $\rlt = 6.2$ realization discussed in the second test, a stationary pattern with basic zonal flow mode $\nzf = 3$ establishes. 
% Does it confirm our previous tests
Again, the $\nzf = 1$ (box scale) zonal flow mode does not grow secularly during the entire turbulent phase.
Also, this test confirms the statistical soundness of the converged pattern size of $\sim 57.20 - 76.27\,\rhoth$.

\includegraphicsHere{Comparison/Gradient-Length/S6_rlt6.0-6.2-6.4_boxsize3x3_Ns16_Nvpar48_Nmu9_eflux_comparison.pdf}{
	Time traces of the heat conduction coefficient $\chi$ for different gradient lengths $\rlt$
}{fig:eflux-6.0-6.2-6.4-comparison}{}