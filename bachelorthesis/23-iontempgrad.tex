\newpage
\section{Ion Temperature Gradient (ITG) Driven Instability}
\label{sec:ITG}

The described electrostatic forces acting on the tokamak plasma causes a variety of collective phenomena. Microinstabilities is one of these phenomena which are instabilities on the spatial scale of the Larmor radius $\rho_\mathrm{L}$ and frequencies much smaller than the cyclotron frequency $\omega_\mathrm{c}$. These instabilities are driven by density and temperature gradients that are present in the tokamak devices. Turbulence caused from microinstabilities are considered as the main reason for the loss of particles energy and momentum from tokamak plasma \cite{Brizard2007, Garbet2010, Horton1999} which limits the confinement time of tokamak reactors. The most dominante instability in tokamak plasma is the ion temperature gradient (ITG) driven instability. \cite{Coppi1967, Cowley1991, Rudakov1961}

\includegraphicsHere{Theory/ITG-Turbulence.pdf}{
	Sketch of ion temperature gradient driven instability on the outboard low field side of the tokamak \cite{Casson_PHD}
}{fig:ITG}{0.5}

The ITG driven instability is present when a sufficiently large ion temperature gradient occurs and is driven by the so-called bad curvature of the outboard low field side of the tokamak. The mechanism will be briefly explained with the help of Fig. \ref{fig:ITG} and Ref. \citenum{Beer_PHD, Casson_PHD, Dannert_PHD}. An ion temperature perturbation ($T^\pm \gtrless 0$ \textcolor{blue}{blue} and \textcolor{red}{red} cells) on the outboard low field side of the tokamak causes the formation of curvature drift and $\nabla B$ dirft with velocity $\vect{v}_\mathrm{d}$ which compresses the plasma ($n_\pm \gtrless 0$, \textcolor{gray}{gray} cells). This relates to the perturbation of the potential ($\phi^\pm  \gtrless 0$) in the adiabatic electron response. As stated in the equation for the electric field ($\vect{E} = -\nabla \phi$) the perturbation of the potential $\phi$ causes an electric field $\vect{E}$ which results in a $\exb$ drift with velocity $\vect{v}_E$. The $\exb$ drifts direction transports hot plasma in the outer hot region of perturbation ($n^+ > 0$, \textcolor{red}{red} cell) and cooler plasma into inner cold region perturbation ($n^- < 0$, \textcolor{blue}{blue} cell) which reinforce the process and instability grows.\bigskip

The described mechanism causes perturbation on the outboard low field side of the tokamak this region has a \enquote{bad curvature} and instability happens there. On the inboard high field side the opposing $\nabla B$ and $\nabla T$ results in a decay of perturbation which is referred to as \enquote{good curvature}. Furthermore, such turbulent structures are called \textit{eddies} which have a typical radial scale of serval Larmor radius. \cite{Newins2006}
Through the explanation above is clear that the initial perturbation is growing when the normalized temperature gradient overcomes a critical value $R/L_\mathrm{T,c}$. For nonlinear turbulence the critical value is given by $R/L_\mathrm{T,c} \sim 6$. \cite{Dimits2000, Isliker2010}