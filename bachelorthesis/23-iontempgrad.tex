\newpage

\section{Ion Temperature Gradient (ITG) Driven Instability}
\label{sec:ITG}

The described electrostatic forces acting on the tokamak plasma causes a variety of collective phenomena. Microinstabilities is one of this phenomena which are instabilities on the spatial scale of the Larmor radius and frequencies much smaller than the cyclotron frequency. These instabilities are driven bei density and temperature gradients that are present in the tokamak devices. Turbulence caused from microinstabilities are considered as the main reason for the loss of particles energy and momentum from tokamak plasma Ref (10,11,12) which limits the confinement time of tokamak reactors. The most dominante instability in tokamak plasma is the ion temperature gradient (ITG) driven instability Ref (13,14,15). 

\includegraphicsHere{Theory/ITG-Turbulence.png}{
	Sketch of ion temperature gradient driven instability on the outboard low field sid of the tokamak
}{fig:ITG}{0.6}

The ITG driven is present when a sufficiently large ion temperature gradient occurs and is driven by the so-called bad curvature of the outboard side of the tokamak. The mechanism will be briefly explained with the help of Fig. \ref{fig:ITG} Ref (18,17,16 and 4 in BA). An ion temperature perturbation ($T^\pm \gtrless 0$ \textcolor{blue}{blue} and \textcolor{red}{red} cells) on the outboard low field side of the tokamak causes the formation of curvature and $\nabla B$ dirft $\vect{v}_\mathrm{d}$ which compresses the plasma ($n_\pm \gtrless 0$, grey cells). This relates to the perturbation of the potential ($\phi^\pm  \gtrless 0$) in the adiabatic electron response. As stated in the equation for the electric field ($\vect{E} = -\nabla \phi$) the perturbation of the potential causes an electric field which results in a $\exb$ drift $\vect{v}_E$. The $\exb$-drifts direction transports hot plasma in the outer hot region of perturbation ($n^+ > 0$, \textcolor{red}{red} cell) and cooler plasma into inner cold region perturbation ($n^- < 0$, \textcolor{blue}{blue} cell) which reinforce the process and instability grows. 

The described mechanism causes perturbation on the outboard side of the tokamak this region has a \enquote{bad curvature} and instability happens there. On the inboard side the opposing $\nabla B$ and $\nabla T$ results in a decay of perturbation which is referred to as \enquote{good curvature}. Furthermore such turbulent structures are called \textit{eddies} and the have a typical radial scale of serval Larmor radius. (Ref 19 BA). \bigskip

Through the explanation above is clear that the initial perturbation is growing when the normalized temperature gradient overcomes a critical value $R/L_\mathrm{T,c}$. For nonlinear turbulence the critical value is given by $R/L_\mathrm{T,c} \sim 6$. (Ref 3, 7 BA)