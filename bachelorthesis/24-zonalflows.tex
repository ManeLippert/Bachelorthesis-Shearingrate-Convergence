\section{Zonal Flows and Shearing Rate $\wexb$}
\label{sec:zonalflow}

Zonal flows are linked in the tokamak plasma to the $\exb$ flows [Eq.(\ref{eq:exb})] that is connected to the on flux-surfaces constant electrostatic potential, also called \textit{zonal potential}. \cite{Diamond2005} Because of the variation exclusively across flux-surfaces, the zonal flow is tangential to those surfaces and does not contribute to the turbulent transport across the flux-surfaces caused by the $\exb$ motion. 
Zonal flows are driven nonlinearly by turbulent Reynolds stresses. \cite{Diamond1991} For more details about the mechanism behind zonal flow generation the reader is referred to Ref. \citenum{Brizard2007, Dubin1983, Hasegawa1978, Krommes2000}. \bigskip

\includegraphicsHere{Theory/Zonal-Flow-Generation.pdf}{
	Subdued turbulence because of stretched eddies, which breaks in smaller structures caused by sheared $\exb$ drifts \cite{Hammett2009}
}{fig:zonalflow}{0.9}

Furthermore, zonal flows are connected with the radially alternating zonal potential, which introduces a velocity shear, i.e., a radial variation of the zonal flow velocity. This causes a shear deformation of turbulent structures which is considered to play a significant role in the suppression of turbulence \cite{Biglari1990, Dimits2000}, especially the ion temperature gradient driven turbulence \cite{Nakata2012, Makwana2014, Maeyama2014, Whelan2018, Whelan2019}. The nonlinear generation mechanism causes a shift, also known as Dimits shift \cite{Dimits2000}, of the critical temperature gradient $R/L_\mathrm{T,c}$ which leads to a later development of ITG driven turbulence. The shearing process tilts and stretches turbulent eddies, caused by the shear flow, until they break into smaller eddies with reduced radial correlation length [Fig. \ref{fig:zonalflow}],  characterizing the shearing as a decorrelation process. Turbulence will then be reduced because of this process. \cite{Biglari1990,Diamond2005,Burnell1997} \bigskip

\newpage

When the zonal flows are strong enough to suppress turbulence completely due to the shearing process, pattern formation occurs in the stabilized tokamak plasma the so-called $\exb$ staircase structure. The $\exb$ staircase pattern has the following properties:
\begin{enumerate}
	\item[(1)] They occur on a \textit{radial mesoscale} of order of 10\,$\rhoth$ which is larger than the Larmor radius but smaller than the machine size $R$.
	\item[(2)] The structure are \textit{quasi stationary} in space and vary on time scales much longer than the typical turbulence time scales.
	\item[(3)] They have a \textit{typical amplitude} in terms of the $\exb$ shearing rate of the order of $10^{-1} \vth/R$.
	\item[(4)] Turbulent transport occurs in \textit{avalanches} with propagation strongly linked to the local $\exb$ shearing rate. \cite{McMillan2009}
\end{enumerate} \bigskip

The $\exb$ staircase pattern is manifest as radial structure formation in the $\exb$ shearing rate defined by\cite{Rath2016, Pueschel2008, Peeters2016}
\begin{equation}
	\wexb = \frac{1}{2} \frac{\partial^2 \langle \phi \rangle}{\partial \xcoord^2}~,
	\label{eq:shearingrate}
\end{equation}
where $\langle \phi \rangle$ is the zonal electrostatic potential normalized by $\rho_\ast T/e$ ($\rho_\ast = \rhoth/R$ is the thermal Larmor radius normalized with the major radius $R$, $T$ is the temperature, $e$ is the elementary charge).
The zonal potential is calculated from the electrostatic potential $\phi$ on the two-dimensional $\xcoord$-$\ycoord$-plane at the low field side according to\cite{Rath2021}
\begin{equation}
\langle \phi \rangle = \frac{1}{L_\ycoord} \int_0^{L_\ycoord} \mathrm{d}\ycoord ~ \phi(\xcoord,\ycoord,s=0)~.
\end{equation}
The $\exb$ shearing rate $\wexb$ is the radial derivative of the advecting zonal flow velocity \cite{Hahm1995, Waltz1998} and quantifies the zonal flow induced shearing of turbulent structures \cite{Biglari1990, Hahm1995, Burnell1997}. \\ \bigskip

As mentioned before the turbulent transport occurs through avalanches which also mediate the turbulence of the ITG driven turbulence in the adabatic electron approximation. Avalanches are initiated when the $\exb$ sharing rate has a zero crossing with a steep flank and propagates outwards with negative shearing and inwards with positive shearing. \cite{Idomura2009,McMillan2009} The fully developed avalanche results in almost completely quenched turbulence. Its development results in a discontinuous step in the dependency of turbulent fluxes on the temperature gradient als known as \textit{finite heat flux threshold}. \cite{Peeters2016,Weikl2017}