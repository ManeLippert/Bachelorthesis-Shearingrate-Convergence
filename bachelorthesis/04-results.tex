% 4. Results


\chapter{Results and Discussion}
\label{chap:results}
\thispagestyle{empty}
\newpage

\section{Variation of Computational Resolution}
\label{sec:variationsofresolution}

At the beginning of this work the goal is to change the used resolutuin to a minimum. The goal behind this research is to reduce Computational time and costs for the computation of the simulation. 

\section{Size Convergence of $\exb$ staircase pattern}
\label{sec:convergence}

This chapter is an further iteration of the brief communication published in \enquote{Physics of Plasma}. It provides additional plots that was not necessary for the publication and some other formatting as well. The brief communication can be find in the appendix of this thesis or under \textbf{REF PAPER GITHUB}
%TODO

\subsection{Radial increased Box Size}
\label{sub:radial}

In the first test the radial box size is increased while the binormal box size is kept fixed to the standard size. The scan covers the realizations: 
\begin{gather*}
	\boxed{\NR \times \NB \in [1\times1,~2\times1,~3\times1,~4\times1]}~.
\end{gather*}
Each realization exhibits an initial quasi-stationary turbulent phase and a second final \cite{peeters2016} phase with almost suppressed turbulence [Fig.~\ref{fig:wexb-eflux-1-2-3-4x1-comparison}\,(a)].
The latter state is indicative for the presence of a fully developed staircase pattern as depicted in Fig.~\ref{fig:wexb-stable-comparison}. 
This type of structure is characterized by intervals of almost constant shear with alternating sign satisfying the Waltz criterion $|\wexb| \approx \gamma$\cite{doi:10.1063/1.870934, doi:10.1063/1.872847} ($\gamma$ is the growth rate of the most unstable linear ITG driven Eigenmode), connected by steep flanks where $\wexb$ crosses zero.
% insofar the results of peeters2016 are well reproduced; now turn to the description of the main result of this test
Fig.~\ref{fig:wexb-stable-comparison}\,(a) shows a striking repetition of the staircase structure, with the number of repetitions equal to $\NR$.
% yes, the radial size converges; but even stronger result: the standard box size already allows for the correct staircase size
Hence, the basic size of the pattern not only converges with increasing radial box size, the converged radial size turns out to at least roughly agree with the standard radial box size of Ref. \citenum{peeters2016}. \bigskip

% cautios words
Due to the lack of a substantial turbulent drive in the final suppressed state no further zonal flow evolution is observed [Fig.~\ref{fig:wexb-eflux-1-2-3-4x1-comparison}\,(b)] and one might critically ask whether the structures shown in Fig.~\ref{fig:wexb-stable-comparison} represent the real converged pattern in a statistical sense. 
% very long initial quasi-stationary states
Note that in the $3 \times 1$ case the initial quasi-stationary turbulent state extends up to a few $\sim 10^4\,R/\vth$.
% the \nzf=3 zonal flow mode does show a temporal evolution with a few cycles covered by the initial turbulent phase -> this allows for some degree of statistical argument
During this period the zonal flow mode with $\nzf = 3$, i.e., the mode that dominates the staircase pattern in final suppressed phase, undergoes a long-term evolution with a typical time scale of several $\sim 10^3\,R/\vth$. 
Hence, several of such cycles are covered by the initial turbulent phase, which is evident from the occurrence of phases with reduced amplitude around $t \approx 8000\,R/\vth$ and $t \approx 18000\,    R/\vth$.
% first important point: competition between the \nzf = 3 and \nzf = 4 mode -> the \nzf = 4 mode is of shorter length scale compared to the \nzf = 3 mode, hence, no convergence to the box scale
It is the $\nzf = 4$ zonal flow mode, i.e., the next shorter radial scale mode, that dominates the shear spectrum $\hatwexbamp$ in the latter two phases (not shown). This demonstrates a competition between the $\nzf = 3$ and $\nzf = 4$ modes.
% second important point: no secular growth of the box scale zonal mode
Most importantly, no secular growth of the $\nzf = 1$ (box scale) zonal flow mode is observed during the entire quasi-stationary turbulent phase [Fig.~\ref{fig:wexb-eflux-1-2-3-4x1-comparison}\,(b) dotted line].
% summary
The above discussion indicates that although the $\nzf = 3,~4$ zonal modes compete, the pattern scale does not converge to the radial box scale but rather to a mesoscale of $\sim 57.20 - 76.27\,\rhoth$ (i.e., $\nzf = 4,~3$ in the $3\times1$ case). \\

\includegraphicsHere{{Comparison/Boxsize/S6_rlt6.0_boxsize1-2-3-4x1_Ns16_Nvpar48_Nmu9_comparison}.pdf}{
	\textbf{(a)} Time traces of the heat conduction coefficient $\chi$ for $\rlt = 6.0$ for radial increased box sizes\\
	\textbf{(b)} Time traces of $\hatwexbamp$ for radial increased box sizes
}{fig:wexb-eflux-1-2-3-4x1-comparison}{}

\newpage
\subsection{Isotropic increased Box Size}
\label{sub:isotropic}

% what kind of analysis
Since the radially elongated simulation domain might inhibit the development of isotropic turbulent structures, in the second test the radial and binormal box size is increased simultaneously.
This scan covers the realizations:
\begin{gather*}
	\boxed{\NR\times\NB \in [1\times1,~2\times2,~3\times3]}~.
\end{gather*}
% mention shorter times scales for stabilization
Interestingly, suppression of the turbulence by the emergence of a fully developed staircase pattern always occurs after $\sim 1000~R/\vth$ [Fig. \ref{fig:eflux-1x1-2x2-3x3-comparison}], i.e., significantly faster compared to the $3\times1$ and $4\times1$ realizations. 
% Although the computational costs per time step is higher, the difference in the time scales renders this test computationally cheaper.
%
% the main result of this test: confirmation of wave length convergence
As shown in Fig.~\ref{fig:wexb-stable-comparison}\,(b) also this test confirms the convergence of the staircase pattern size to a typical mesoscale that is distinct from the radial box size in the $\NR > 1$ realizations.

\includegraphicsHere{{Comparison/Boxsize/S6_rlt6.0_boxsize1x1-2x2-3x3_Ns16_Nvpar48_Nmu9_comparison}.pdf}{
	Time traces of the heat conduction coefficient $\chi$ for $\rlt = 6.0$ for isotropic increased box sizes
}{fig:eflux-1x1-2x2-3x3-comparison}{}

% discuss sumewhat smaller radial size
By contrast to the radial box size scan the $3\times3$ realization shows a stationary pattern with four repetitions of the fully developed staircase structure, i.e., a somewhat smaller pattern size. 
Whether this is related to a possible pattern size dependence on the binormal box size or to the competition between patterns with the two sizes $\lambda \in [57.20,~ 76.27]\,\rhoth$ as observed in the first test is addressed in the next paragraph.

\newpage
\subsection{Binormal increased Box Size}
\label{sub:binormal}

% what kind of analysis
In a third test the binormal box size is varied with the radial box size fixed to $\NR = 3$.
This test covers the realizations:
\begin{gather*}
	\boxed{\NR \times \NB \in [3\times1.5,~3\times2.5,~3\times3,~3\times5]}~.
\end{gather*}
As in the isotropic scan the turbulence subdued and a fully developed staircase pattern forms after $\sim 2000\,R/\vth$ [Fig.~\ref{fig:eflux-3x1.5-2.5-3-5-comparison}]. The convergence of staircase pattern can be seen in Fig.~\ref{fig:wexb-stable-comparison}\,(c) and confirms again a size of a typical mesoscale. Fig.~\ref{fig:wexb-stable-comparison}\,(c) also confirms that indeed a competition between patterns with two sizes $\lambda \in [57.20,~ 76.27]\,\rhoth$ causing the different results for $3 \times 1$ and $3\times 3$. The zonal flow mode number varies between $\nzf = 3,4$ which can be seen in Fig.~\ref{fig:wexb-stable-comparison}\,(c) in the $3\times 2.5$ realization. The staircase structure has a pattern between $3$ and $4$ repetitions which get represented in the second repetition with no signifciant plateau at positive shear. Instead the pattern returns immediately after reaching the maximum shear ($+ \gamma$) to the minimum shear ($- \gamma$) of the third repetition in a steep flank. The Fourier analysis of this case yields no definitely basic mode rather two dominating modes with $\nzf = 3, 4$ with a fraction of the maximum amplitude $\hatwexbamp$ each (not shown).
\includegraphicsHere{{Comparison/Boxsize/S6_rlt6.0_boxsize3x1-1.5-2.5-3-5_Ns16_Nvpar48_Nmu9_comparison}.pdf}{
	Time traces of the heat conduction coefficient $\chi$ for $\rlt = 6.0$ for binormal increased box sizes
}{fig:eflux-3x1.5-2.5-3-5-comparison}{}\bigskip

\subsection{Staircase structures in Comparison}
\label{sub:comparisonwexb}

\includegraphicsHere{{Comparison/Boxsize/S6_rlt6.0_boxsize1-2-3-4x1-1.5-2-2.5-3-5_Ns16_Nvpar48_Nmu9_wexb_comparison}.pdf}{
	Comparison of shearing rate $\wexb$ for each box sizes scan averaged over given time interval and the growth rate $\pm \gamma$ of the most unstable linear ITG driven Eigenmode. The staircase structures are radially shifted with respect to each over till alignment for better visibility.\\
	\begin{tabular}{l l l l l}
		\textbf{(a) radial:}    & $t_{1\times 1}$   & $\in [2000, 5000]$,   & $t_{2\times 1}$   & $\in [15000, 18000]$, \\
		                        & $t_{3\times 1}$   & $\in [43000, 45000]$, & $t_{4\times 1}$   & $\in [26000, 28000]$  \\
		\textbf{(b) isotropic:} & $t_{1\times 1}$   & $\in [2000, 5000]$,   & $t_{2\times 2}$   & $\in [2000, 3000]$,   \\
		                        & $t_{3\times 3}$   & $\in [2000, 3000]$    &                   &                       \\
		\textbf{(c) binormal:}  & $t_{3\times 1.5}$ & $\in [2000, 3000]$,   & $t_{3\times 2.5}$ & $\in [2000, 3000]$,   \\
		                        & $t_{3\times 3}$   & $\in [2000, 3000]$    & $t_{3\times 5}$   & $\in [1000, 3000]$    \\
	\end{tabular}
}{fig:wexb-stable-comparison}{}

\section{The finite heat flux threshold}
\label{sub:threshold}

In the final test the inverse background temperature gradient length $\rlt$ is varied at fixed $3\times3$ box size.
Since suppression of turbulence usually occurs at later times when approaching the finite heat flux threshold from below \cite{peeters2016}, the analysis aims to lengthen the phase during which the zonal flow varies in time due to turbulent Reynolds stresses.
This scan covers realizations with:
\begin{gather*}
	\boxed{\rlt \in [6.0,~6.2,~6.4]}~.
\end{gather*}
% what is the finite heat flux threshold
In the case of $\rlt = 6.2$ turbulence suppression is observed for $t > 11000\,R/\vth$, while stationary turbulence during the entire simulation time trace of $12000\,R/\vth$ is found for $\rlt = 6.4$.
The finite heat flux threshold, hence, is:
\begin{gather*}
	\boxed{\rlt|_\mathrm{finite} = 6.3 \pm 0.1}
\end{gather*}
in accordance to Ref. \citenum{peeters2016}.
% Which stationay radial size occurs????
Although the initial quasi-stationary turbulence in the former case is significantly longer compared to the $\rlt = 6.2$ realization discussed in the second test, a stationary pattern with basic zonal flow mode $\nzf = 3$ establishes. 
% Does it confirm our previous tests
Again, the $\nzf = 1$ (box scale) zonal flow mode does not grow secularly during the entire turbulent phase.
Also, this test confirms the statistical soundness of the converged pattern size of $\sim 57.20 - 76.27\,\rhoth$.\\