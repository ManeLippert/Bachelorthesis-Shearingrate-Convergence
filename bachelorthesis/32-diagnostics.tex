\newpage
\section{Diagnostics}
\label{sec:diagnostics}

In order to diagnose the temporal evolution of the staircase pattern and to obtain an estimate of its amplitude the radial Fourier transform of the $\exb$ shearing rate is considered. 
It is defined by
\begin{equation}
	\wexb = \sum_{\kzf} \hatwexb(\kzf,t) \, \exp(\mathrm{i} \kzf \xcoord)~,
	\label{eq:shearingrate_fourier}
\end{equation}
where $\hatwexb$ is the complex Fourier coefficient and $\kzf = 2\pi \nzf/L_\xcoord$
% explicitly introduce the terms zonal flow wave vector and zonal flow mode number for later usage
defines the zonal flow wave vector with the zonal flow mode number $\nzf$ ranging in $-(N_\xcoord -1)/2 \leq \nzf \leq (N_\xcoord -1)/2 $ and the radial box size $L_\xcoord$.
% define the zonal flow modes' amplitude in terms of the shearing rate
Based on the definitions above, the shear carried by the zonal flow mode with wave vector $\kzf$ is defined by $\hatwexbamp = 2 |\hatwexb(\kzf,t)|$. 
% introduces the term basic mode of the pattern for later usage
In general, the zonal flow mode that dominates the $\exb$ staircase pattern, also referred to as the \textit{basic mode} of the pattern in this work, exhibits the maximum amplitude in the spectrum $\hatwexbamp$.