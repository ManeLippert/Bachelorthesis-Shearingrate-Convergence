\newpage
\section{Magnetic Confinement in Tokamak}
\label{sec:confinement}

In tokamak devices strong magnetic fields confine the hot plasma. As mentioned in Chapter \ref{sec:motion} a magnetic field forces a perpendicular particle motion and a motion which contains the gyro motion and slow perpendicular gyro center drifts. Because of the much smaller size of the Larmor radius compared to the device size $R$ the particle and energy losses are caused by the gyro center drift. To avoid this at the end of the field lines the magnetic field in the tokamak devices is shaped like a torus. This type of geometry has nested surfaces with constant magnetic flux, so-called \textit{flux surfaces}, and magnetic field lines which appear on these surfaces. To maintain stability the magnetic field has a toroidal and a poloidal component which also is equivalant to the plasma pressure. \cite{Stroth2011, Wesson2011} The toroidal component is produced by external coils whereas the poloidal component is provided by the toroidal plasma current. Together the components result in a magnetic field which follows helical trajectories [Fig \ref{fig:confinement}]. To characterize the quality of confinement the so-called \textit{plasma beta} is used and is given as
\begin{gather}
    \beta = \frac{nT}{\mu_0 B^2/2}~,
\end{gather} 
with $n$ the plasma density, $T$ as temperature, $\mu_0$ the permeability in vacuum and the magnetic field strength $B$. Respectively, the plasma beta compares the thermal plasma pressure $nT$ to the ambient magnetic field pressure $\mu_0 B^2/2$ which indicates that in cases when the plasma beta is smaller the confinement is better. For fusion devices the plasma beta has to be smaller than 1 ($\beta < 1$). In a tokamak reactor the plasma beta has a typical order of a few percent. \cite{Wesson2011}
\includegraphicsHere{Theory/Tokamak-Torus.pdf}{
	Toroidal flux surfaces in tokamak plasma with helical magnetic field (\textcolor{Ubtgreen}{green} line) in torus coordinates ($\rho$ (radial), $\phi$ (toroidal), $\theta$ (poloidal)) or cylindrical coordinates ($Z$, $R$, $\phi$). \cite{Barton2015}
}{fig:confinement}{0.7}