\chapter{Velocity Nonlinearity and Conservation Properties}
\section{The velocity nonlinearity} 
\label{sec:velocity-nonlinearity} 
\index{Velocity nonlinearity} 

There is a nonlinearity called the velocity nonlinearity, which is often neglected in the $\delta f$ formalism since it is one order smaller 
in the normalized Larmor radius $\rho_*$ compared with the leading order terms. The
expression for the velocity nonlinearity can be obtained by considering the evolution equation for 
of the parallel velocity (Here without rotation and only electro-static) 
\begin{equation}
mv_\parallel \d{v_\parallel}{t} = - \d{\bf X}{t}\cdot \left [
Ze \nabla \langle \phi \rangle + \mu \nabla B \right ] 
\end{equation}
In the $\delta f$ formalism the terms that are proportional to both the field $\phi$ as well as the perturbed 
distribution $f$ are neglected. 
Note that these terms do include the term due to ${\bf v}_E \cdot \mu \nabla B$, which cancels 
${\bf v}_{\nabla B} \cdot Z e \nabla \langle \phi \rangle$. The remaining terms are 
\begin{equation}
 \d{v_\parallel}{t} = - \frac{Ze }{ m v_\parallel} [ v_\parallel {\bf b} + {\bf v}_{D-} ] \cdot 
\nabla \langle \phi \rangle  
\end{equation}
where ${\bf v}_{D-}$ is the drift velocity minus the grad-B drift. 

The parallel velocity nonlinearity therefore adds an additional term to the gyro-kinetic equation in the 
form 
\begin{equation}
\pd{f}{ t} \overset{+}{=} \frac{Ze }{ m v_\parallel} [ v_\parallel {\bf b} + {\bf v}_{D-} ] \cdot 
\nabla \langle \phi \rangle  \pd{ f }{v_\parallel} 
\end{equation}
Normalizing this contribution gives for the parallel motion and the drift, respectively 
\begin{equation}
\pd{f_N}{t_N} \overset{+}{=} \frac{Z \rho_*}{2 \sqrt{T_{GN} m_N}}
 {\cal F} \pd{ \langle \phi_N \rangle }{s} 
\pd{ f_N }{v_{\parallel N}} 
\end{equation}
\begin{equation}
\pd{ f_N}{t_N} \overset{+}{=}  
\frac{\rho_*^2 }{ 2} \left [ v_{\parallel N} {\cal D}^\alpha + v_{\parallel N} \beta^\prime {\cal E}^{1 \alpha} + 
2 H^\alpha \sqrt{\frac{m_N }{ T_{GN}}} \Omega_N \right ] 
\pd{ \langle \phi_N \rangle}{ x^\alpha} \pd{ f_N }{v_{\parallel N}} 
\label{eq:parallel-velocity-nonlinearity-drift}
\end{equation}
(here the Coriolis drift is kept which is somewhat inconsistent with the fact that it has been neglected in the 
equation above, but it is work in progress). 
The drift terms involve derivatives in the perpendicular plane that scale as $1/\rho_*$. Therefore, both 
terms are of the order $\rho_*$ and are usually neglected. 

As discussed in section \ref{energy-conservation-local-limit}, in the case of the local limit the velocity nonlinearity 
must be implemented as 
\begin{equation}
\pd{ f}{ t} \overset{+}{=} \pd{}{ v_\parallel} \left [ \frac{Ze }{ m v_\parallel} 
[ v_\parallel {\bf b} + {\bf v}_{D-} ] \cdot \nabla \langle \phi \rangle  f \right ]  
\end{equation}
in order to assure phase space convervation. In the latter case an additional term appears in Eq.~(\ref{eq:parallel-velocity-nonlinearity-drift})
which must be implemented as 
\begin{align}
\pd{ f_N}{ t_N} &\overset{+}{=}
\frac{\rho_*^2 }{ 2} \left [ v_{\parallel N} {\cal D}^\alpha + v_{\parallel N} \beta^\prime {\cal E}^{1 \alpha} + 
2 H^\alpha \sqrt{\frac{m_N}{T_{GN}}} \Omega_N \right ] 
\pd{ \langle \phi_N \rangle}{ x^\alpha} \pd{ f_N }{v_{\parallel N}} \cr 
\noalign{\vskip 0.2 truecm} \\
& + \frac{\rho_*^2 }{ 2} \left [ {\cal D}^\alpha + \beta^\prime {\cal E}^{1 \alpha} \right ] \pd{ \langle \phi \rangle }{x^\alpha} f 
\end{align}

\section{Basic equation} 

{\bf Warning: not all is correct in this section} 

The equations in GKW are related to, but do not directly follow from a Lagrangian description. 
Further approximations to the Lagrange equations of motion are made in order to obtain a system that 
is easier to solve numerically. The approximations all involve neglecting higher order $\rho_*$ terms,
and since the Lagrangian from which the equations are derived itself is accurate up to terms linear in $\rho_*$ 
the neglection of higher order terms is consistent with accuracy of the Lagrange description. 
Care has to be taken not to lose the properties that come with the Lagrange description, like conservation 
of energy and momentum. Below it will be shown that these properties still hold for the model equations 
used by GKW. 

Starting point of the derivation is the system Lagrangian \index{Lagrangian} 
\begin{equation}
L = \sum_{sp} \int {\rm d}^3{\bf X} {\rm d}^3 {\bf v} \, f L_p + \sum_{sp} \int {\rm d}^3 {\bf X} 
{\rm d}^3 {\bf v} \, \frac{Z_{sp}^2 e^2}{2 T_{sp} } [ \phi^2 - \langle \phi \rangle^2 ] F_M + \sigma_a \int {\rm d}^3 {\bf X} \,  L_{\rm adiabatic} 
+ \int {\rm d}^3 {\bf X} \, \frac{1 }{ 2 \mu_0} \vert \nabla A_\parallel \vert^2 
  \label{eq:lagrangian}
\end{equation}
with the particle-Lagrange density 
\begin{equation}
L_p = \Gamma_a \dot z^a -H  = \left ( Z e {\bf A} + m {\bf u}_0 +  m v_\parallel {\bf b} + Z e \langle A_\parallel \rangle  {\bf b} \right ) \cdot \d{\bf X}{t} 
- \left ( \frac{1 }{ 2} m v_\parallel^2 + \mu B + Z e \langle \phi \rangle + \frac{1 }{ 2} m u_0^2 \right ) ,
\label{eq:plagrangian}
\end{equation}
where $L_p$ is written without the species index. The quantity $\Gamma$ refers to the discussion in section~\ref{sec:theoryLagrangian}, eq.~\eqref{eq:gyrocenterlagrangian}.
The expressions above describe a gyrocenter lagrangian in the sense that the rapid gyro-motion has been removed from the Lagrangian. The indices in the Einstein 
summation convention are $a = 1,2,3,4$ with $a = 4$ refering to the parallel velocity (Note that $\gamma_4$ is zero though). 
The brackets $\langle \rangle$ correspond to the gyro-average operator, and the sum with index sp is over all 
kinetic species. 
The vector potential ${\bf A}$ describes the (constant) background magnetic field, while the perturbed magnetic field is 
described by $A_\parallel$, the perturbed parallel component of the vector potential. 

In the case of adiabatic electrons, the electron contribution is not included in the species-sum $\sum_{sp}$ but is treated by $\sigma_a = 1$ in a special term in eq. \eqref{eq:lagrangian}.
(If electrons are meant to be kinetic, $\sigma_a = 0$ takes that term out.)
The Lagrangian density associated with adiabatic electrons in that term is 
\begin{equation}
L_{\rm adiabatic} = - \frac{e^2 n_e }{ 2 T_e} [ \phi - \{ \phi \} ]^2 ,  
\end{equation}
where $\{ \}$ indicates the flux surface average. 

\nomenclature[1L]{$L$}{The Lagrangian}
\nomenclature[1Lp]{$L_p$}{The particle Lagrangian}
\nomenclature[1Z]{$Z$}{The charge number}
\nomenclature[1X]{${\bf X}$}{The gyro-centre spatial coordinate}
\nomenclature[1v]{${\bf v}$}{The velocity} 
\nomenclature[1f]{$f$}{The perturbed distribution} 
\nomenclature[1e]{$e$}{The unit charge}
\nomenclature[1T]{$T$}{The temperature} 
\nomenclature[2f]{$\phi$}{The perturbed potential} 
\nomenclature[3b]{$\langle \rangle $}{The gyro-average operation} 
\nomenclature[3c]{$\langle \rangle^\dagger$}{The conjugate transpose of the gyro-average operator} 
\nomenclature[1fm]{$F_M$}{The Maxwell distribution} 
\nomenclature[1H]{$H$}{The Hamiltonian} 

In the following, We occasionally split the Hamiltonian $H$ from \eqref{eq:plagrangian} into a field part $H_1 = Z e \langle \phi \rangle$ and the rest $H_0$ (which is time independent),
\begin{align}
  H &= \underbrace{\frac{1 }{ 2} m v_\parallel^2 + \mu B +  \frac{1 }{ 2} m u_0^2}_{H_0} +\underbrace{Z e \langle \phi \rangle}_{H_1} = H_0(\mathbf{X}, v_\parallel, \mu) + H_1(\mathbf{X}, t, \mu) 
\label{eq:Hsplit}
\end{align}

Variation of $L$, which is already integrated over phase space, towards $\phi$
\begin{align}
  \frac{\delta L}{\delta \phi} &\overset{!}{=} 0  \\
   &= - \sum_{sp}\int\ud^3v 
\frac{\delta}{\delta\phi}
\left(\int\ud^3X f_{s,tot}
    \left[\underbrace{Ze\ga{\phi}}_{H_1} + \frac{Z_s^2e^2}{2T_s}\left(\ga{\phi}^2 - \phi^2\right)\right]
 \right)
 - \sigma_a\frac{\delta}{\delta\phi} \int \ud\mathbf{X} \frac{e^2 n_e }{ 2 T_e} [ \phi - \{ \phi \} ]^2
\label{eq:Lfuncderivphi}
\end{align}
yields the Poisson equation \index{Poisson equation}.
\ifmoredetails
To see how, consider the derivative of the first term in \eqref{eq:Lfuncderivphi}, which becomes
\begin{align}
  &\d{}{\epsilon}\left[\int\ud^3X^\prime f_{tot}(\mathbf{X}^\prime) Ze\ga{\phi(\mathbf{X}^\prime) + \epsilon\delta(\mathbf{X}^\prime - \mathbf{X})}\right]_{\epsilon=0}  \nonumber\\
{}={}&\d{}{\epsilon}\left[\int\ud^3X^\prime \iga{ f_{tot}}(\mathbf{X}^\prime,\mu) Ze(\phi(\mathbf{X}^\prime) + \epsilon\delta(\mathbf{X}^\prime - \mathbf{X}))\right]_{\epsilon=0}  \nonumber\\
{}={}& \int\ud^3X^\prime \iga{f_{tot}}(\mathbf{X}^\prime,\mu) Ze \delta(\mathbf{X}^\prime - \mathbf{X}) \nonumber\\
{}={}& Ze \iga{ f_{tot}}(\mathbf{X},\mu) 
\label{eq:Lfuncderivphi1stterm}
\end{align}
Here, the gyroaverage operator was written as $\ga{}$ and must be distinguished from its adjoint $\iga{}$.
The second term of \eqref{eq:Lfuncderivphi} becomes
  \begin{align}
   &\d{}{\epsilon}\Big[\int\ud^3X^\prime F_M\ga{\phi + \epsilon\delta(\mathbf{X}^\prime - \mathbf{X})}^2 &&- (\phi + \epsilon\delta(\mathbf{X}' - \mathbf{X}))^2\Big]_{\epsilon=0} \nonumber\\
   {}={}& \int\ud^3X^\prime F_M 2\ga{\phi + \epsilon\delta(\mathbf{X}^\prime - \mathbf{X})}\ga{\delta(\mathbf{X}^\prime - \mathbf{X})}\Bigg|_{\epsilon=0} &&+ 2F_M\phi(\mathbf{X})\nonumber\\
   {}={}& \int\ud^3X^\prime F_M 2\ga{\phi}\ga{\delta(\mathbf{X}^\prime - \mathbf{X})}&&+ 2F_M\phi(\mathbf{X})\nonumber\\
   {}={}& \int\ud^3X^\prime \iga{F_M 2\ga{\phi}}\delta(\mathbf{X}^\prime - \mathbf{X})&&+ 2F_M\phi(\mathbf{X})\nonumber\\
   {}={}&  F_M 2\iga{\ga{\phi}}(\mathbf{X},\mu)+ 2F_M\phi(\mathbf{X}) \label{eq:Lfuncderivphi2ndterm}
   \end{align}
and the third one works analogously to the simpler part in the right column of \eqref{eq:Lfuncderivphi2ndterm}.
Derived this way, the Poission equation is then
\fi
\begin{equation}
\sum_{sp} Z_{sp} e \int {\rm d}^3 {\bf v} \, \langle f \rangle^\dagger -\sigma_a \frac{n_e e^2 }{ T_e} ( \phi - \{ \phi \} ) 
= \sum_{sp} \frac{Z_{sp}^2 e^2}{T_{sp}} \int {\rm d}^3 {\bf v} \, [ \phi - \langle \langle \phi \rangle \rangle^\dagger ] F_{Msp} 
\label{eq:poissonequation}
\end{equation}
Furthermore, variation towards $A_\parallel$ yields Ampere's law. Again, we work each term separately:
\begin{align}
 \frac{\delta L}{\delta A_\parallel} &\overset{!}{=} 0  \\  
&= \frac{\delta}{\delta A_\parallel}
 \sum_{sp} \int {\rm d}^3{\bf X} {\rm d}^3 {\bf v} \, f Ze\ga{A_\parallel}\mathbf{b}
+ \frac{\delta}{\delta A_\parallel}\int {\rm d}^3 {\bf X} \, \frac{1 }{ 2 \mu_0} \vert \nabla A_\parallel \vert^2 
\end{align}
The first term is evaluated just like \eqref{eq:Lfuncderivphi1stterm}.
\begin{align}
   \frac{\delta}{\delta A_\parallel}
 \sum_{sp} \int {\rm d}^3{\bf X} {\rm d}^3 {\bf v} \, f Ze\ga{A_\parallel}\mathbf{b}\cdot\d{\mathbf{X}}{t}
&= 
\sum_{sp}\int\ud^3\mathbf{v}
\d{}{\epsilon}\left[\int\ud^3X^\prime \f_{tot}(\mathbf{X}^\prime) Ze\ga{A_\parallel(\mathbf{X}^\prime) + \epsilon\delta(\mathbf{X}^\prime - \mathbf{X})}\mathbf{b}\cdot\d{\mathbf{X}}{t}\right]_{\epsilon=0}  \nonumber\\
&= 
\sum_{sp}\int\ud^3\mathbf{v}
Ze\iga{f_{tot}\mathbf{b}\cdot\d{\mathbf{X}}{t}}(\mathbf{X},\mu)\\
&= 
\sum_{sp}\int\ud^3\mathbf{v}
Ze\iga{f_{tot}}(\mathbf{X},\mu)\mathbf{b}\cdot\d{\mathbf{X}}{t}
\end{align}
The second term is already more interesting due to the gradient.
\begin{align}
  \frac{\delta}{\delta A_\parallel}\int {\rm d}^3 {\bf X} \, \frac{1 }{ 2 \mu_0} \vert \nabla A_\parallel \vert^2 
&=\frac{1}{2\mu_0}\d{}{\epsilon}\left[\int\ud^3X^\prime |\nabla \left(A_\parallel(\mathbf{X}^\prime)+ \epsilon\delta(\mathbf{X}^\prime - \mathbf{X})\right))|^2\right]_{\epsilon=0}  \nonumber\\
&=\frac{1}{2\mu_0}\d{}{\epsilon}\left[\int\ud^3X^\prime (\nabla \left(A_\parallel(\mathbf{X}^\prime)+ \epsilon\delta(\mathbf{X}^\prime - \mathbf{X})\right))^2\right]_{\epsilon=0}  \nonumber\\
&=\frac{1}{2\mu_0}2\left[\int\ud^3X^\prime (\nabla\left( A_\parallel(\mathbf{X}^\prime)+ \epsilon\delta(\mathbf{X}^\prime - \mathbf{X})\right))\nabla\delta(\mathbf{X}^\prime - \mathbf{X})\right]_{\epsilon=0}  \nonumber\\
&=\frac{1}{\mu_0}\int\ud^3X^\prime (\nabla A_\parallel(\mathbf{X}^\prime))\nabla\delta(\mathbf{X}^\prime - \mathbf{X})  \nonumber\\
&= \underbrace{\frac{1}{\mu_0}\int_\partial\ud^2X'(\nabla A_\parallel(\mathbf{X}^\prime))\delta(\mathbf{X}^\prime - \mathbf{X})}_0
- \frac{1}{\mu_0}\int\ud^3X^\prime (\nabla^2 A_\parallel(\mathbf{X}^\prime))\delta(\mathbf{X}^\prime - \mathbf{X})  \nonumber\\
&= -\frac{1}{\mu_0}\nabla^2A_\parallel
\end{align}
%FIXME SRG how does the boundary term vanish? For a periodic box, ok, but generally?
That is, this variation yields Ampere's law.
\begin{equation}
- \frac{1}{\mu_0}\nabla^2 A_\parallel = \sum_{sp} Z_{sp} e \int {\rm d}^3 {\bf v} \, \d{\bf X}{t} \cdot {\bf b} \langle f \rangle^\dagger  
\label{eq:ampereslaw}
\end{equation} 

GKW uses the Poisson equation and Ampere's law above, i.e. they follow directly from the Lagrange description. 
Approximations, however, are made in the equations of motion of the particles. The Lagrange description yields the 
equations of motion through the variation of the action 
\begin{equation}
\delta S = \delta \int_{t_0}^{t_1} L_p {\rm d} t  
= \int_{t_0}^{t_1} \pd{ \gamma_a}{ z^b} \delta z_b \dot z^a + \gamma_a \delta \dot z^a - 
\pd{ H}{ z^a} \delta z^a {\rm d} t . 
\label{eq:eqmotionderivation}
\end{equation}
Integrating the term with $\delta \dot z^a$ once partial towards the time yields
\ifmoredetails
\begin{align}
\delta S = \int_{t_0}^{t_1}  \gamma_a \delta \dot z^a  {\rm d} t 
&=  \gamma_a \delta z^a\Big|_{t_0}^{t_1} - \int_{t_0}^{t_1} \dot\gamma_a \delta z^a  {\rm d} t \\
&=  - \int_{t_0}^{t_1} \Big(\underbrace{\pd{\gamma_a}{z^b}}_{\gamma_{a,b}}\pd{z^b}{t} - \pd{\gamma_a}{t}\Big)\delta z^a  {\rm d} t
\end{align}
and rename $a\leftrightarrow b$ in the first term of \eqref{eq:eqmotionderivation}, to get
\fi
  the equations of motion in the form 
\begin{align} 
[ \gamma_{b,a} - \gamma_{a,b} ] \d{z^b}{ t} &= \pd{ H}{ z^a} + \pd{ \gamma_a}{ t} \label{Lagrangemotion} 
\end{align}
The equation above can be split into components $(1,2,3)$ and the fourth component as follows 
\begin{align}
\Gamma = \left ( Z e {\bf A} + m {\bf u}_0 +  m v_\parallel {\bf b} + Z e \langle A_\parallel \rangle  {\bf b} \right )  
\end{align}
\begin{align}
(\nabla \times {\bf A}^*) \times \d{\bf X}{t} + \pd{{\bf A}^*}{v_\parallel} \d{v_\parallel}{t} 
&= - \nabla H - \pd{{\bf A}^*}{t} \label{eqmotion} &&\text{eq. of motion for $v_\parallel$}\\
\pd{{\bf A}^*}{v_\parallel} \cdot \d{\bf X}{t} &= \pd{H}{v_\parallel} \label{eqmotion2} &&\text{eq. of motion for position space coordinates }
\end{align}
where 
\begin{align}
Z e {\bf A}^* &= Z e {\bf A} + {\bf u}_0 + m v_\parallel {\bf b} + A_\parallel {\bf b} 
\label{eq:Astar}
\\
{\bf B}^* = \nabla \times {\bf A}^*
\end{align}
% FIXME so $Ze\mathbf{A}^* \equiv \mathbf{\Gamma}$ from \eqref{eq:plagrangian}?
% FIXME is there a gyroavg. around the $A_\parallel$ or not? in \eqref{eq:plagrangian}?

The equations \eqref{eqmotion} and \eqref{eqmotion2} above determine both the evolution of both the position as well as the parallel velocity of a gyro-centre, in given fields. 
A set of reduced equations can be derived by taking the inner product of the first equation \eqref{eqmotion} with $\,{{\rm d} {\bf X} / {\rm d} t}$ to 
obtain 
\begin{align}
0 + \left (\d{\bf X}{t} \cdot  \pd{{\bf A}^*}{v_\parallel} \right ) \d{v_\parallel}{t} 
&= -  \d{\bf X}{t} \cdot \nabla H -  \d{\bf X}{t} \cdot \pd{{\bf A}^*}{t} 
\label{motionred1}
\\
 \pd{{\bf A}^*}{v_\parallel} \cdot \d{\bf X}{t} 
&= \pd{H}{v_\parallel} 
\label{motionred2} 
\end{align}
and by inserting \eqref{motionred2} into \eqref{motionred1} 
\begin{align}
\pd{H}{v_\parallel} \d{v_\parallel}{t} 
+  \d{\bf X}{t} \cdot \nabla H 
&= 
 -  \d{\bf X}{t} \cdot \pd{{\bf A}^*}{t} 
\label{motionred3}
\end{align}


The essential observation here is that these equations will guarantee
energy conservation independent of the choice of
${{\rm d} {\bf X} / {\rm d} t}$. To proof this we will use that
\begin{align}
%\d{J_v(t,\mathbf{X}, v_\parallel, \mu)}{t} &= 0 \qquad |\cdot \frac{1 }{ J_v} 
%\label{eq:phasespacecons}
%\\
\frac{1 }{ J_v} \pd{J_v}{ t} + \frac{1 }{ J_v} (\nabla \cdot  J_v) \d{\bf X}{t} 
 + \frac{1 }{ J_v}\pd{J_v}{ v_\parallel}  \d{v_\parallel}{ t} 
 + \frac{1 }{ J_v}\pd{J_v}{ \mu}  \underbrace{\d{\mu}{t}}_0
&= 0,
\label{eq:phasespacecons2}
\end{align}
where $J_v$ is the Jacobian of the velocity space integration. 
%FIXME I do not really understand how the second equation is constructed. Is it really from the first line?|
This condition \eqref{eq:phasespacecons2} expresses conservation of the gyrocenter phase space volume of an element in the flow.
While this condition is perhaps not under all circumstances necessary to obtain
exact energy conservation, it is in our derivation an additional constraint 
that must be satisfied for the gyrocentre velocity.  
This observation allows one to make approximations to ${{\rm d}{\bf X}/ {\rm d}t}$, removing higher order $\rho_*$ corrections, without 
breaking energy conservation. 

Adding the equations \eqref{motionred3} and \eqref{eq:phasespacecons2} above we have 
\begin{align}
\pd{ H}{ v_\parallel} \d{v_\parallel}{t}
+  \d{\bf X}{ t} \cdot \nabla H 
&+&H
%\left
(
\frac{1 }{ J_v} \pd{J_v}{ t} 
+ \frac{1 }{ J_v} \nabla \cdot  [ J_v \d{\bf X}{ t}  ] 
+ \frac{1 }{ J_v}\d{}{ v_\parallel}  [ J_v \d{v_\parallel}{ t}  ] 
%\right
)
&=
- \d{\bf X}{t} \cdot \pd{ {\bf A}^*}{ t} 
\end{align}
\begin{align}
\pd{H}{v_\parallel} \d{v_\parallel}{t} 
+ H \frac{1 }{ J_v} \underbrace{\pd{J_v}{ t}}_0
+ H \frac{1 }{ J_v}\pd{}{ v_\parallel} \left [ J_v \d{v_\parallel}{ t} \right ]
&+& 
  \d{\bf X}{t} \cdot \nabla H 
+ H \frac{1 }{ J_v} \nabla \cdot \left [ J_v \d{\bf X}{t} \right ]
&= 
-  \d{\bf X}{t} \cdot \pd{{\bf A}^*}{t} 
\\
\frac{1 }{ J_v} \pd{}{ v_\parallel} \left [J_v H \d{v_\parallel}{t} \right ] 
&+&
\frac{1 }{ J_v} \nabla \cdot \left [ J_v H \d{\bf X}{t} \right ] 
&= - \d{\bf X}{t} \cdot \pd{{\bf A}^*}{t} 
\end{align}

Then we split the Hamiltonian \eqref{eq:Hsplit} and use that the field part $H_1 = Z e \langle \phi \rangle$ is time dependent, whereas the rest $H_0$ is not.
\begin{align}
 \d{H}{t} &= \d{H_0}{t} + \d{H_1}{t} \\
%\pd{H}{v_\parallel}\d{v_\parallel}{t} + \nabla H \d{X}{t} + \cancel{\pd{H}{t}} &= \pd{H_0}{v_\parallel}\d{v_\parallel}{t} + \nabla H_0 \d{X}{t} + \nabla H_1 \d{X}{t} + \cancel{\pd{H_1}{t}}
\pd{H}{v_\parallel}\d{v_\parallel}{t} + \nabla H \d{X}{t} + \pd{H}{t} &= \pd{H_0}{v_\parallel}\d{v_\parallel}{t} + \nabla H_0 \d{X}{t} + \nabla H_1 \d{X}{t} + \pd{H_1}{t}
\pd{H}{v_\parallel}\d{v_\parallel}{t} + \nabla H \d{X}{t}&= \pd{H_0}{v_\parallel}\d{v_\parallel}{t} + \nabla H_0 \d{X}{t} + \nabla H_1 \d{X}{t} 
\end{align}
to obtain 
\begin{align} 
\d{ H_0 }{t} = 
\frac{1 }{ J_v} \pd{}{ v_\parallel} \left [J_v H_0 \d{v_\parallel}{t} \right ] + 
\frac{1 }{ J_v} \nabla \cdot \left [ J_v H_0 \d{\bf X}{t} \right ] 
&= - \d{\bf X}{t} \cdot \left [  \nabla H_1 + Z e \pd{{\bf A}^*}{t} \right ]  
\end{align}
Using furthermore that $\pd{f}{t} = 0$ we arrive at an energy balance equation. 
\begin{equation} 
\d{ }{ t} \left [ H_0 f \right ] 
= -  \d{\bf X}{t} \cdot \left [  \nabla H_1 + Z e \pd{{\bf A}^*}{t} \right ] f 
\label{eq:energyconsParticles}
\end{equation} 
The left hand side of this equation represents the change of kinetic energy.
The right hand side represents the inner product of
current and electric field. It is the transfer term between kinetic
and field energy.

We now would like to investigate the transferm term in more detail by deriving the corresponding energy balance from the field equations.
We consider first the Poisson equation, afterwards the Ampere equation.
Multiplying the time derivative of the Poisson equation \eqref{eq:Lfuncderivphi} with $\phi$ we obtain 

\begin{align}
0 &= - \phi \pd{}{t}\sum_{sp}\int\ud^3v 
\frac{\delta}{\delta\phi}
\left(\int\ud^3X f_{s,tot}
    \left[\underbrace{Ze\ga{\phi}}_{H_1} + \frac{Z_s^2e^2}{2T_s}\left(\ga{\phi}^2 - \phi^2\right)\right]
 \right)
 - \sigma_a\phi \pd{}{t}\frac{\delta}{\delta\phi} \int \ud\mathbf{X} L_a
\end{align}
The functional derivative of the first term can be executed, and the rest is put to the right hand side.
We may also substract the neutrality equation for the background $F_{sp,M}$ and continue with $f_{sp}$ in place of $f_{sp,tot} = F_{sp,M} + f_{sp}$.
\begin{align}
\sum_{sp} Z_{sp} e \phi \pd{}{t} \int {\rm d}^3 {\bf v} \, \langle f \rangle^\dagger 
&=
 % \sum_{sp} \frac{Z_{sp}^2 e^2}{T_{sp}}\phi\pd{}{t} \int {\rm d}^3 {\bf v} \, [ \phi - \langle \langle \phi \rangle \rangle^\dagger ] F_{Msp} 
% +
\phi \pd{}{t}\frac{\delta}{\delta\phi}\sum_{sp}\int\ud^3v \int\ud^3X f_{s,tot}\frac{Z_s^2e^2}{2T_s}\left(\ga{\phi}^2 - \phi^2\right)
+
\sigma_a\phi \pd{}{t}\frac{\delta}{\delta\phi} \int \ud\mathbf{X} L_a
\end{align}
If we consider only quantities which are first order in $\rho_*$ then $f_{s,tot}$ is only $F_{s,M}$ and the fluctuation part of the distribution is neglegted for this term.
\begin{align}
 \sum_{sp}\int {\rm d}^3 {\bf v} \, Ze\phi \pd{\iga{f}}{t} 
&=  \phi \pd{}{ t} \left [ \frac{\delta H_{2E}}{\delta \phi} \right ] 
\end{align} 
where 
\begin{equation} 
H_{2E} = \sum_{sp}\int {\rm d}^3 {\bf v} \int\ud\mathbf{X}\, \frac{Z_{sp}^2 e^2}{2 T_{sp} } [ \phi^2 - \langle \phi \rangle^2 ] F_M + \sigma_a \int\ud\mathbf{X}L_{adiabatic} 
\end{equation} 
By putting both sides under a position space integral $\int\ud^3\mathbf{X}$ again, we can roll the gyroaverage over to $\phi$ and identify $H_1$.
\begin{align}
 \sum_{sp} \int {\rm d}^3 {\bf X} \int {\rm d}^3 {\bf v} \, H_1 \pd{ f}{ t} 
&=\int{\rm d}^3 {\bf X} \phi \pd{}{ t} \left [ \frac{\delta H_{2E}}{\delta \phi} \right ] 
\end{align}
And using ${\rm d} f / {\rm d} t = 0$ the first term in the equation can be rewritten in the form 
\begin{equation} 
- \sum_{sp}\int {\rm d}^3 {\bf X} \int {\rm d}^3 {\bf v} \, \d{\bf X}{t} \cdot \nabla H_1\, f  
= \int{\rm d}^3 {\bf X} \phi \pd{}{ t} \left [ \frac{\delta H_{2E}}{\delta \phi} \right ] 
\label{eq:energyconsPoisson}
\end{equation} 
%FIXME SRG I dont see how getting from $\pd{}{t}$ to $\d{}{t}$ is supposed to work.|
On the left hand side of this equation we can already identify a part of the energy transfer term.
Lets also evaluate the right hand side and construct time derivatives of quadratic quantities.
\begin{align}
- \sum_{sp}\int {\rm d}^3 {\bf X} \int {\rm d}^3 {\bf v} \, \d{\bf X}{t} \cdot \nabla H_1\, f  
= &
\int\ud\mathbf{X}
\phi \pd{}{ t}
\Big(
\sum_{sp}
 \int {\rm d}^3 {\bf v} \, \frac{Z_{sp}^2 e^2}{T_{sp}} [ \phi - \iga{\ga{\phi}} ] F_{M,sp}+{} \nonumber\\
&{}+ \sigma_a \frac{n_e e^2 }{ T_e} (\phi - \{ \phi \} ) 
\Big)\\
=& 
\int\ud\mathbf{X}
\pd{}{ t}
\Big(
\sum_{sp}
 \int {\rm d}^3 {\bf v} \, \frac{Z_{sp}^2 e^2}{T_{sp}} \frac{1}{2}[ \phi^2 - \ga{\phi}^2 ] F_{M,sp}
\Big)
+{} \nonumber\\
&{}+ \sigma_a \int\ud\mathbf{X}\frac{n_e e^2 }{ T_e} (\frac{1}{2}\pd{\phi^2}{t} - \phi\pd{}{t}\{ \phi \} ) 
\end{align} 
In order to arrive at the form of \eqref{eq:energyconservation}, we
rewrite the term with the flux surface averaged potential $\fsa{\phi}$
as follows. We take advantage of the fact that taking a flux surface avg. is executed by
integrating over $s$ like $\fsa{.} = \int . \ud s$, and taking further averages has no effect $\fsa{\fsa{\phi}}= \fsa{\phi}$.
\begin{align}
\int\ud s\frac{1}{2}\pd{}{t}(\phi - \fsa{\phi})^2  
&=\fsa{\frac{1}{2}2(\phi-\fsa{\phi})\pd{}{t}(\phi-\fsa{\phi})}
\\
&=\fsa{\phi\pd{}{t}\phi - \phi\pd{}{t}\fsa{\phi}-\fsa{\phi}\pd{}{t}\phi + \fsa{\phi}\pd{}{t}\fsa{\phi}
}
\\
&=\fsa{\phi\pd{}{t}\phi - \phi\pd{}{t}\fsa{\phi}}
-\fsa{\phi}\pd{}{t}\fsa{\phi} + \fsa{\fsa{\phi}\pd{}{t}\fsa{\phi}}
\\
&=\int \pd{}{t}\frac{1}{2}\phi^2-\phi\pd{}{t}\fsa{\phi} \ud s
\end{align}

We proceed analogously for Ampere's law \eqref{eq:ampereslaw} by multiplying it with $ A_\parallel\pd{}{t}$ and integrating over the whole of position space. 
\begin{align}
 \int\ud\mathbf{X}A_\parallel\pd{}{t}
 \sum_{sp} Z_{sp} e \int {\rm d}^3 {\bf v} \, \d{\bf X}{t} \cdot {\bf b} \langle f \rangle^\dagger  
&=
 -\int\ud\mathbf{X}A_\parallel\pd{}{t}\frac{1}{\mu_0} \nabla^2 A_\parallel
\\
\sum_{sp} Z_{sp} e\int\ud\mathbf{X}\ud^3\mathbf{v}\ga{A_\parallel}\pd{}{t}
   \d{\bf X}{t} \cdot {\bf b}  f 
&=
 -\frac{1}{\mu_0}\int_\partial\ud^2\mathbf{X} A_\parallel\nabla A_\parallel + \frac{1}{\mu_0} \int\ud\mathbf{X}(\nabla A_\parallel)\pd{}{t}\nabla A_\parallel
\end{align}
% FIXME SRG I dont see how to find $\mathbf{A}^*$ here.
% FIXME SRG There must be a gyroavg around some $A$ in the energy balance, too, somewhere.
We obtain 
\begin{equation} 
\sum_{sp} Z_{sp} e \int {\rm d}^3 {\bf X} {\rm d}^3 {\bf v} \,\d{\bf X}{t} \cdot \pd{{\bf A}^*}{t}f 
=
 \pd{}{t}\frac{1 }{ 2 \mu_0} \int {\rm d}^3 {\bf X} \, \vert \nabla A_\parallel \vert^2   
\label{eq:energyconsAmpere}
\end{equation}
Consequently the energy conservation theorem results from adding
\eqref{eq:energyconsParticles}, \eqref{eq:energyconsPoisson} and \eqref{eq:energyconsAmpere}, which yields
\index{Noether's theorem, Energy conservation}  
\begin{align}
  \d{}{t}\int\ud^3\mathbf{X}\ud^3\mathbf{v}H_0f 
&= - \int\ud^3\mathbf{X}\ud^3\mathbf{v} \d{\bf X}{t} \cdot \left [  \nabla H_1 + Z e \pd{{\bf A}^*}{t } \right ] f 
\\
&= \int{\rm d}^3 {\bf X} \phi \pd{}{ t} \left [ \frac{\delta H_{2E}}{\delta \phi} \right ] +  \frac{1 }{ 2 \mu_0} \int {\rm d}^3 {\bf X} \, \vert A_\parallel \vert^2   
\\
\pd{E}{t} &= 0\\
E &=  \sum_{sp} \int {\rm d}^3 {\bf X} {\rm d}^3 {\bf v} \, \left ( \frac{1 }{ 2} m v_\parallel^2 + \mu B  + \frac{1 }{ 2} m u_0^2 \right ) f + \nonumber{}\\
&+
\sum_{sp} \int {\rm d}^3 {\bf X} {\rm d}^3 {\bf v} \, \frac{Z_{sp}^2 e^2}{2 T_{sp} } [ \phi^2 - \langle \phi \rangle^2 ] 
F_{Msp} + \sigma_a \int {\rm d}^3 {\bf X} \, \frac{n_e e^2 }{ 2 T_e} (\phi - \{ \phi \} )^2 + \frac{1 }{ 2 \mu_0} \int {\rm d}^3 {\bf X} \, \vert \nabla A_\parallel \vert^2 
\label{eq:energyconservation}
\end{align}
where the first part is the kinetic energy while the second part is the field energy. The Noether theorem on the original Maxwellian 
gives exactly the same energy theorem. Furthermore, note that no assumption has been made on the form of ${\rm d} {\bf X} / {\rm d} t$ other 
than that the equations and ?? must be satisfied. 

Taking the cross product of eq.~(\ref{eqmotion}) with the unit vector along the magnetic field $\mathbf{b}$ one obtains the velocity of the gyro-centre 
\begin{align} 
\mathbf{b}\times\left(\mathbf{B}^* \times \d{\bf X}{t} + \pd{{\bf A}^*}{v_\parallel} \d{v_\parallel}{t} \right)
&= - \mathbf{b}\times\left(\nabla H - \pd{{\bf A}^*}{t}\right)
\\
\mathbf{B}^*(\mathbf{b}\cdot\d{\bf X}{t}) - \d{\bf X}{t}\underbrace{(\mathbf{b}\cdot\mathbf{B}^*)}_{B^*_\parallel}
+ \mathbf{b}\times\pd{{\bf A}^*}{v_\parallel} \d{v_\parallel}{t} 
&= - \mathbf{b}\times\left(\nabla H - \pd{{\bf A}^*}{t}\right)
\\
\d{ {\bf X} }{ t}
&= 
\frac{\mathbf{B}^*}{B^*_\parallel}(\mathbf{b}\cdot\d{ {\bf X} }{ t})
 + \frac{\mathbf{b}}{B^*_\parallel}\times\nabla H 
+ \frac{\mathbf{b}}{B^*_\parallel}\times\underbrace{\pd{\mathbf{A}^*}{v_\parallel}}_{m\mathbf{b}} \d{v_\parallel}{t} 
- \frac{\mathbf{b}}{B^*_\parallel}\times\underbrace{\pd{\mathbf{A}^*}{t}}_{\pd{A_\parallel}{t}\mathbf{b}}
\\
\d{\bf X}{t} &= \frac{{\bf B}^*}{B^*_{\parallel}} v_\parallel + \frac{{\bf b}}{B^*_\parallel} \times \nabla H 
\label{eq:gyrocentrevelocity}
\end{align}
where
\begin{equation} 
B_\parallel^* = {\bf b} \cdot {\bf B}^* = {\bf b} \cdot \nabla \times {\bf A}^* 
\label{eq:Bparallelstar}
\end{equation}
and $J_v = B_\parallel^*$. 
It was also used that the parts of $\mathbf{A}^*$ which are dependent of $v_\parallel$ and $t$ are parallel to $\mathbf{b}$, cf \eqref{eq:Astar}.

When used in the form of \eqref{eq:Bparallelstar}, $B_\parallel^*$ is a function of the field $\langle A_\parallel \rangle$, and therefore of time. 
Since ${\rm d} {\bf X} / {\rm d} t$ depends nonlinearly on $B_\parallel^*(t)$ this complicates the numerical solution. At each 
time point the gyro-centre velocity would have to be constructed explicitly and the various terms in the equation can not 
be precalculated and stored in matrix format. 

The approximation made in GKW is the approximation of $B_\parallel^*$ by $B$. Working out ${\bf B}^*$ 
with the Hamiltonian from \eqref{eq:Hsplit}
we then obtain 
\begin{align} 
\d{{\bf X}}{t}  &\approx 
\frac{{\bf B}^* }{ B} v_\parallel 
+ \frac{{\bf b}}{ B} \times \nabla H\\
% \left( \frac{1}{2} m v_\parallel^2 + \mu B +  \frac{1 }{ 2} m u_0^2 +Z e \langle \phi \rangle\right) 
&=
\frac{1}{B} v_\parallel \nabla\times\left({\bf A} + \frac{1}{Ze}{\bf u}_0 + \frac{m v_\parallel}{Ze} {\bf b} + \frac{1}{Ze}A_\parallel {\bf b} \right)
+ \frac{{\bf b}}{ B} \times \nabla H
\\
&=v_\parallel {\bf b} + \frac{m v_\parallel^2}{ZeB} \nabla \times {\bf b} + \frac{m v_\parallel}{ZeB} {\bf \Omega}
 +  \frac{v_\parallel}{B} \nabla \times [ \langle A_\parallel \rangle {\bf b} ] 
+ \frac{1 }{ B} {\bf b} \times \nabla H 
\label{eq:gyrocentrevelocity2}
\end{align}
%FIXME where $\mathbf{\Omega} = \frac{1}{m}\nabla\times\mathbf{u}_0$?
With the approximation $J_v = B^*_\parallel \approx B$ it can be shown that eq.~\eqref{eq:phasespacecons2} is satisfied.
%FIXME SRG What is $J_v$?

A few remarks should be made here. First, the approximation
\begin{equation} 
\nabla \times \ga{ A_\parallel} {\bf b}
= 
\nabla \ga{A_\parallel}\times\mathbf{b} + \ga{A_\parallel}\nabla\times\mathbf{b}
 \approx - {\bf b} \times \nabla \langle A_\parallel \rangle 
\label{eq:approxtofindExBdrift2}
\end{equation}
is used in \eqref{eq:approxtofindExBdrift} to split the gyrocentre velocity into grad-B and ExB drifts.
This approximation is problematic for exact energy conservation since in this case eq.~\eqref{eq:phasespacecons2} is not satisfied. 
%FIXME how does this enter \eqref{eq:phasespacecons2}?|

\subsection{Local limit} 

\label{energy-conservation-local-limit}
A brief look ahead for the local limit approximation may be in order. The local limit corresponds to a further approximation also in the gyro-centre velocity and must be 
separately discussed. 
Essentially, the local limit is the limit where the scale length of perpendicular dynamics, which is of the order of the Larmor radius $\rho_\tref$, is much much smaller than the equilibrium scale length $R_\tref$.
\begin{align}
  \rho_\tref \ll R_\tref
\end{align}
that means
\begin{align}
  \rho_* = \frac{\rho_\tref}{R_\tref} \ll 1
\end{align}
This implies further assumptions:
\begin{itemize}
\item all background and geometry quantities are assumed to be a function of the parallel coordinate ($s$) only.
\item periodic poundary conditions
\end{itemize}

The gyro-centre velocity \eqref{eq:gyrocentrevelocity2} does not depend on the coordinates perpendicular to the field and, therefore the divergence of this velocity is different from the expression derived above. 
% FIXME ?? Sure it does, there is $\nabla\ga{\phi}$, and this has dependencies on perp coords in leading order?!
% FIXME Which divergence expression above? There is none!?
The gyrocentre velocity \eqref{eq:gyrocentrevelocity2} can be
understood to consist of parallel motion and several drift motions.
\begin{align}
\d{\mathbf{X}}{t}
&=
v_\parallel {\bf b} + \frac{m v_\parallel^2 }{ ZeB} \nabla \times {\bf b} + \frac{m v_\parallel }{ ZeB} {\bf \Omega}
+ \frac{v_\parallel }{ B} \nabla \times [ \langle A_\parallel \rangle {\bf b} ] 
+ \frac{1 }{ B} {\bf b} \times \nabla \left ( \frac{1 }{ 2} m v_\parallel^2 + \mu B  + \frac{1 }{ 2} m u_0^2 
+ Z e \ga{\phi}
\right ) ,
\\
&= v_\parallel\mathbf{b}  
+ {\bf v}_D 
+  \underbrace{\frac{v_\parallel }{ B} \nabla \times [ \langle A_\parallel \rangle {\bf b} ]  + \frac{1 }{ B} {\bf b} \times \nabla Z e \langle \phi \rangle}_{\mathbf{v}_\chi}
\end{align}
where $\mathbf{v}_D$ is a perpendicular drift velocity due to the background magnetic fields and reference frame rotation. A drift of the form of an ExB drift can be identified if the additional approximation \eqref{eq:approxtofindExBdrift} is made:
\begin{equation} 
{\bf v}_\chi = \frac{{\bf b} \times \nabla \chi}{B} \qquad 
\textrm{with} \qquad \chi = \langle \phi \rangle - v_\parallel \langle A_\parallel \rangle ,
\end{equation}
As mentioned above at \eqref{eq:approxtofindExBdrift2}, this approximation makes the system violate phase space conservation, though.

Nevertheless, in the local limit, i.e. the limit of small $\rho_*$, phase space conservation is valid again:
In this limit (i.e. to lowest order in $\rho_*$) the drift velocity ${\bf v}_D$ and ExB velocity ${\bf v}_E$ do not have a component in the direction of $\mathbf{e}^s\equiv\nabla s$, i.e. ${\bf v}_D\cdot \nabla s = v_D^s = {\bf v}_E \cdot \nabla s =v_E^s= 0$. 
This is also strictly imposed in the code when the finite $\rho_*$ parallel derivatives are turned off. 

With the velocity space jacobian
\begin{align}
  J_v \propto B
\end{align}
%FIXME I do not find an expression for the velspace jacobian anywhere.
we consider
\begin{align}
B\d{\mathbf{X}}{t}\cdot\nabla f = B ( v_\parallel {\bf b} + {\bf v}_D + {\bf v}_E ) \cdot \nabla f 
&=
\nabla \cdot [ B ( v_\parallel {\bf b} + {\bf v}_D + {\bf v}_E ) f ] - f\nabla \cdot [ B ( v_\parallel {\bf b} + {\bf v}_D + {\bf v}_E )] 
\\
&=
 ... - f\nabla B(s)\cdot ( v_\parallel {\bf b} + {\bf v}_D + {\bf v}_E ) - f B(s)\nabla\cdot ( v_\parallel {\bf b} + {\bf v}_D + {\bf v}_E )
% \\
% &=
%  ... - f(\mathbf{e}^s\pd{}{s} + \mathbf{e}^\psi\pd{}{\psi} + \mathbf{e}^\zeta\pd{}{\zeta} ) \cdot [ B(s) ( v_\parallel {\bf b} + {\bf v}_D + {\bf v}_E )]
\\
&=
 ... - f(\nabla s\pd{B(s)}{s}) \cdot ( v_\parallel {\bf b} + {\bf v}_D + {\bf v}_E )
\\
&\qquad\qquad
 - fB(s)\frac{1}{J}\left(
\pd{J(v_\parallel {\bf b} + {\bf v}_D)^s}{s} + \pd{J({\bf v}_E)^\psi}{\psi} + \pd{J({\bf v}_E)^\zeta}{\zeta} 
\right) 
\\
&=
 ... - f\pd{B(s)}{s} v_\parallel \underbrace{\mathbf{e}^s\cdot\frac{\mathbf{B^s\mathbf{e}_s}}{B}}_{0}
\\
&\qquad\qquad
 - fB(s)\frac{1}{J}\left(
\pd{J(v_\parallel {\bf b} + {\bf v}_D + \mathbf{v}_E)^s}{s} + \pd{J({\bf v}_E)^\psi}{\psi} + \pd{J({\bf v}_E)^\zeta}{\zeta} 
\right) 
\\
&=
 - fB(s)\frac{1}{J}\left(
\pd{Jv_\parallel \frac{\mathbf{B^s\mathbf{e_s}}}{B}\cdot\mathbf{e}^s}{s} + \pd{J({\bf v}_E)^\psi}{\psi} + \pd{J({\bf v}_E)^\zeta}{\zeta} 
\right) 
\\
&=
 - fB(s)\frac{1}{J}\left(
\pd{J({\bf v}_E)^\psi}{\psi} + \pd{J({\bf v}_E)^\zeta}{\zeta} 
\right) 
\\
&= ? FIXME
\\
&= \nabla \cdot [ B ( v_\parallel {\bf b} + {\bf v}_D + {\bf v}_E ) f ]
\end{align}
i.e. the rewrite above shows that the whole gyro-centre velocity satisfies 
\begin{equation}
%\frac{1 }{ B}
 \nabla \cdot \left [ B \d{\bf X}{t} \right ] \propto \nabla \cdot \left [ J_v \d{\bf X}{t} \right ] = 0
\end{equation}
The phase space conservation \eqref{eq:phasespacecons2} is then reduced to
\begin{align}
  \frac{1 }{ J_v} \pd{J_v}{ t} 
+ \frac{1 }{ J_v} (\nabla \cdot  J_v) \d{\bf X}{t} 
 + \frac{1 }{ J_v}\pd{J_v}{ v_\parallel}  \d{v_\parallel}{ t} 
 &= 0,
\end{align}
Then 
\begin{equation} 
\pd{}{ v_\parallel } \left [ \d{v_\parallel}{t} \right ] \quad \rightarrow \quad 
\pd{}{ v_\parallel } \left [ - \frac{1 }{ m} {\bf b} \cdot \mu \nabla B \right ] = 0 
\end{equation}
must be zero in order for phase space conservation to apply. 
% FIXME why? how does this enter the phasespacecons?
% FIXME free floating term here? What is this supposed to tell?
% FIXME Rico writes one has to implement 
% \begin{align}
%   \pd{}{v_\parallel}\left[\d{v_\parallel}{t}f\right]
% \end{align}
% rather than
% \begin{align}
%   \d{v_\parallel}{t}\pd{f}{v_\parallel}
% \end{align}
% but this is not the same as the term left of the arrow.
% FIXME nobody says why, I suppose this is to implement a conservative
% method - as far as I understand the matter (and the book of Toro on
% Riemand Solvers for CFD), however, (1) a conservative method only
% helps to conserve the quantitity the equation conserves ($f$ here) and
% (2) if the solution is sufficiently smooth (i.e. no shock
% discontinuities) it does not matter.


The terms on the right (the trapping term) represent the only term of ${\rm d} v_\parallel / {\rm d} t$ that is kept for the perturbed distribution $f$. 
Indeed for this term the parallel velocity derivative is zero and, consequently the phase space volume is conserved in the evolution of $f$. 

The local limit is obtained through an expansion in $\rho_*$, and neglection of all terms which are of order $\rho_*^2$ or higher. In doing so, the velocity nonlinearity is neglected.
%FIXME What is the precise expression for the velocity nonlinearity?
A more correct approach 
%(FIXME cite Scott) 
would be to neglect contribution already at the level of the Lagrangian and make no approximation thereafter. This would ensure energetic consistency between particle and field dynamics.
Following such a calculation, it turns out that neglection of certain nonlinear terms (the ``velocity nonlinearity'') is not allowed, i.e. they must be kept in the model to achieve energy conservation although they are formally $\orderof (\rho_*^2)$ terms in the particle dynamics equations.

In fact, in the local limit with periodic boundary conditions,
conservation of the energy \eqref{eq:energyconservation} is
problematic since the total field energy is always positive, whereas
the total kinetic energy
\begin{equation} 
\int {\rm d}^3 {\bf X} \int {\rm d}^3 {\bf v} \, \left [ \frac{1 }{ 2} mv_\parallel^2 + \mu B \right ] f 
\end{equation} 
integrates to zero for all Fourier modes except the mode that has a zero wave vector in both perpendicular directions (below this mode is refered to as the $(0,0)$ mode). 
The $(0,0)$ mode is often not kept in the evolution equation (particularly in linear
 simulations), and it is then not possible to conserve the energy since the positive 
field energy can not be balanced with a 
negative total kinetic energy. 

Thus, it is necessary to keep both the $(0,0)$ mode
and the velocity nonlinearity in the evolution equations, in order to be able to achieve energy conservation (there are further obstacles though).
It is beneficial to implement  the velocity nonlinearity in the form of \eqref{eq:vel-nonlin-implementation} rather than \eqref{eq:vel-nonlin-not-implementation}.
\begin{align} 
\pd{f}{t} &\overset{+}{=} -\pd{}{v_\parallel} \left [ \d{v_\parallel}{t} f \right ] 
\label{eq:vel-nonlin-implementation}
\\
&= -\d{}{t}(\pd{v_\parallel}{v_\parallel}) f - \d{v_\parallel}{t} \pd{f}{v_\parallel}\nonumber\\
&= - \d{v_\parallel}{t} \pd{f}{v_\parallel}
\label{eq:vel-nonlin-not-implementation}
\end{align}
The form \eqref{eq:vel-nonlin-implementation} ensures phase space conservation in the local limit where the divergence of the gyro-centre velocity is zero. 
In the equation above the change in the parallel velocity is considered only through the electro-magnetic field. 
It can easily be shown that when integrated over the whole computational domain the term only contributes to the $(0,0)$ mode. 
The $(0,0)$ mode has no spatial derivatives perpendicular to the field, and except through the velocity nonlinearity it does not couple to any other mode. 
The $(0,0)$ mode therefore only acts as a reservoir for the kinetic energy essuring that the balance equation for the energy holds. 

%FIXME this sounds as if it was a success. According to Rico it is not - but Rico does not show data either!

\subsection{Toroidal momentum conservation} 

The equations of motion (Eq.~(\ref{Lagrangemotion}) can be written in the form \eqref{eq:Lagrangemotion-other-form}.
\begin{align} 
[ \gamma_{b,a} - \gamma_{a,b} ] \d{z^b}{ t} &= \pd{ H}{ z^a} + \pd{ \gamma_a}{ t}
\nonumber\\
\gamma_{b,a}\d{z^b}{ t} - \gamma_{a,b} \d{z^b}{ t} &= \pd{ H}{ z^a} + \pd{ \gamma_a}{ t}
\nonumber\\
 - \pd{ \gamma_a}{ t} - \gamma_{a,b} \d{z^b}{ t} &= \pd{ H}{ z^a} - \gamma_{b,a}\d{z^b}{ t}
\nonumber\\
\pd{ \gamma_a}{ t} + \gamma_{a,b} \d{z^b}{ t} &= -\pd{ H}{ z^a} + \gamma_{b,a}\d{z^b}{ t}
\nonumber\\
\d{\gamma_a}{t} &= - \pd{H}{z^a} - \gamma_{b,a} \d{z^b}{t} 
\label{eq:Lagrangemotion-other-form}
\end{align} 
%FIXME I get a different sign here, + instead of the - that was here before.
To have a symmetry in coordinate $z^a$ means that Hamiltonian $H$ and $\gamma_b$ are independent of $z^a$.
\begin{equation} 
\pd{H}{z^a} = 0 \qquad 
\gamma_{b,a} = 0 
\end{equation} 
and from \eqref{eq:Lagrangemotion-other-form} follows then
\begin{equation} 
\d{ \gamma_a }{ t} = 0 
\end{equation}
This represents a conservation equation, in the sense that $\gamma_a$ is constant along along the trajectory of a particle.

The case of interest here is the toroidal symmetry of the tokamak (and the consequent 
conservation of toroidal angular momentum). To discuss this, note that both $\langle \phi \rangle$ and 
$\langle A_\parallel \rangle$ are a function of the toroidal angle $\varphi$, but background quantities such as $\mathbf{A}$, $\mathbf{u_0}$, $\mathbf{b}$, and $H_0$ (cf. \eqref{eq:Hsplit}) are not.
Choosing the toroidal angle $\varphi$ as coordinate one then
obtains from \eqref{eq:Lagrangemotion-other-form}
\begin{align} 
\d{\gamma_\varphi}{t} &= - \pd{H_1}{\varphi} + Ze \pd{\langle A_\parallel \rangle}{\varphi } \mathbf{b}\cdot \d{\bf X}{t} 
\nonumber\\
&= - \pd{H_1}{\varphi} + Ze \pd{\langle A_\parallel \rangle}{\varphi } v_\parallel \\
\d{\gamma_\varphi}{t} &= - \pd{H_1}{\varphi} - Ze \pd{\langle A_\parallel \rangle}{\varphi } v_\parallel \d{\bf X}{t} 
\end{align}
%FIXME Mistake in second term here, right?|
Multiplying this equation with $f$ and integrating over the whole phase space (using ${{\rm d} f / {\rm d} t} = 0$ we obtain 
\begin{align} 
\int {\rm d}^3 {\bf X} {\rm d}^3\d{\gamma_\varphi}{t}f &= - \int {\rm d}^3 {\bf X} {\rm d}^3\mathbf{v}\left [
\pd{H_1}{\varphi} - Ze \pd{\langle A_\parallel \rangle}{\varphi } v_\parallel
 \right]f
\\
 \d{}{ t} \int {\rm d}^3 {\bf X} {\rm d}^3 {\bf v} \, \gamma_\varphi f&= - \int {\rm d}^3 {\bf X} {\rm d}^3\mathbf{v}\left [
\pd{H_1}{\varphi} - Ze \pd{\langle A_\parallel \rangle}{\varphi } v_\parallel
 \right]f
\\
\pd{}{ t} \int {\rm d}^3 {\bf X} {\rm d}^3 {\bf v} \, \gamma_\varphi f &= 
- \int {\rm d}^3 {\bf X} {\rm d}^3 {\bf v} \,\left [ \pd{H_1}{\varphi} + Z e \pd{\langle A_\parallel\rangle}{ \varphi} v_\parallel \right ] f 
\end{align}
The right hand side of this equation can be shown to vanish
%FIXME text missing?| 

%%% Local Variables:
%%% mode: latex
%%% TeX-master: "doc"
%%% End:
