% AUTHOR ===================================================================================================================
% Manuel Lippert (GitHub: ManeLippert (https://github.com/ManeLippert))
% ==========================================================================================================================

% PREAMBLE =================================================================================================================

% ****** Start of file aipsamp.tex ******
%
%   This file is part of the AIP files in the AIP distribution for REVTeX 4.
%   Version 4.1 of REVTeX, October 2009
%
%   Copyright (c) 2009 American Institute of Physics.
%
%   See the AIP README file for restrictions and more information.
%
% TeX'ing this file requires that you have AMS-LaTeX 2.0 installed
% as well as the rest of the prerequisites for REVTeX 4.1
% 
% It also requires running BibTeX. The commands are as follows:
%
%  1)  latex  aipsamp
%  2)  bibtex aipsamp
%  3)  latex  aipsamp
%  4)  latex  aipsamp
%
% Use this file as a source of example code for your aip document.
% Use the file aiptemplate.tex as a template for your document.

% DOCUMENT =================================================================================================================

\documentclass[aip, amsmath, amssymb, reprint, twocolumn, floatfix]{revtex4-1}

% documentclass revtex4-1 options:
% - aip,
% - jmp,
% - bmf,
% - sd,
% - rsi,
% - amsmath,amssymb,
% - preprint,%
% - reprint,%
% - author-year,%
% - author-numerical,%
% - Conference Proceedings

\preprint{AIP/123-QED}

% PACKAGES =================================================================================================================

\usepackage[utf8]{inputenc}
\usepackage[T1]{fontenc}
\usepackage{mathptmx}
\usepackage{etoolbox}
\usepackage{graphicx}% Include figure files
\usepackage{dcolumn}% Align table columns on decimal point
\usepackage{bm}% bold math
%\usepackage[mathlines]{lineno}% Enable numbering of text and display math
%\linenumbers\relax % Commence numbering lines

%% Additional
\usepackage[format=hang]{caption}
\usepackage{hyperref}

% FUNCTIONS, INPUT =========================================================================================================

% AUTHOR ==============================================================================================================
% Manuel Lippert (GitHub: ManeLippert (https://github.com/ManeLippert))
% =====================================================================================================================

\usepackage{xparse}
\usepackage{booktabs}
\usepackage{ifthen}
\usepackage{xcolor}
\usepackage{refcount}

% COUNTER, WORD COUNT, VARIABLE =======================================================================================

\def\wordlimit{3500}
\def\pagelimit{4}

\newcounter{totalwordcounter}
\newcounter{remainwordcounter}
\addtocounter{remainwordcounter}{\wordlimit}
\newcounter{wordcounter}

% Figures
\newcounter{figOneCol}      \newcommand{\numberforafigOneCol}{200}
\newcounter{figTwoCol}      \newcommand{\numberforafigTwoCol}{400}

% Tables
\newcounter{tableOneCol}    \newcommand{\numberforatableOneCol}{13}
\newcounter{tableOneColRow} \newcommand{\numberforatableOneColRow}{5}
\newcounter{tableTwoCol}    \newcommand{\numberforatableTwoCol}{26}
\newcounter{tableTwoColRow} \newcommand{\numberforatableTwoColRow}{13}

% Equations
\newcounter{eqOneColRow}    \newcommand{\numberforaeqOneColRow}{7}
\newcounter{eqTwoColRow}    \newcommand{\numberforaeqTwoColRow}{13}

% FUNCTIONS ===========================================================================================================

% Count words in document
\newread\somefile
\makeatletter
\NewDocumentCommand{\wordcount}{s}{%
  \immediate\write18{texcount -sum -1 \jobname.tex > wordcount.txt}%
  \immediate\openin\somefile=wordcount.txt%
  \read\somefile to \@@localdummy%
  \immediate\closein\somefile%
  \setcounter{wordcounter}{\@@localdummy}%
  \IfBooleanF{#1}{%
  \@@localdummy%   print only if not starred version
  }%
}
\makeatother

% Total word count and colors output
\newcommand{\totalwordcount}{
    % Words
    \addtocounter{totalwordcounter}{\value{wordcounter}}
    % Figures
    \addtocounter{totalwordcounter}{\number\numexpr\numberforafigOneCol*\value{figOneCol}}
    \addtocounter{totalwordcounter}{\number\numexpr\numberforafigTwoCol*\value{figTwoCol}}
    % Tables
    \addtocounter{totalwordcounter}{\number\numexpr\numberforatableOneColRow*\value{tableOneColRow}}
    \addtocounter{totalwordcounter}{\number\numexpr\numberforatableOneCol*\value{tableOneCol}}
    \addtocounter{totalwordcounter}{\number\numexpr\numberforatableTwoColRow*\value{tableTwoColRow}}
    \addtocounter{totalwordcounter}{\number\numexpr\numberforatableTwoCol*\value{tableTwoCol}}
    % Equations
    \addtocounter{totalwordcounter}{\number\numexpr\numberforaeqOneColRow*\value{eqOneColRow}}
    \addtocounter{totalwordcounter}{\number\numexpr\numberforaeqTwoColRow*\value{eqTwoColRow}}

    %\number\value{totalwordcounter}
    %\renewcommand{\totalwordcount}{\number\value{totalwordcounter}}

    % Color Output
    \ifnum\value{totalwordcounter}>\wordlimit
        \textcolor{red}{\showcounter{totalwordcounter}}
    \else
        \textcolor{teal}{\showcounter{totalwordcounter}}
    \fi
}

% Display value of counter
\newcommand{\showcounter}[1]{\number\value{#1}}

% Subtract counters
\newcommand{\subtocounter}[2]{\setcounter{#1}{\numexpr\value{#1}-\value{#2}}}

\newcommand{\increasecounter}[3]{
    \ifthenelse{\equal{#2}{1}}{\def\col{OneCol}}{\def\col{TwoCol}}

    \ifthenelse{\equal{#1}{fig}}{\addtocounter{#1\col}{1}}
        {\ifthenelse{\equal{#1}{eq}}{\addtocounter{#1\col Row}{#3}}
            {\addtocounter{#1\col}{1} \addtocounter{#1\col Row}{#3}}
        }
}

% Color output for page numbers
\newcommand{\totalpagecount}{
    \ifnum\getpagerefnumber{LastPage}>\pagelimit
        \textcolor{red}{\pageref*{LastPage}}
    \else
        \textcolor{teal}{\pageref*{LastPage}}
    \fi
}

% Color output for remain counter
\newcommand{\remainwordcount}{

    \subtocounter{remainwordcounter}{totalwordcounter}

    \ifnum\value{remainwordcounter}<0
        \textcolor{red}{\showcounter{remainwordcounter}}
    \else
        \textcolor{teal}{\showcounter{remainwordcounter}}
    \fi
}

% Display word count message
\newcommand{\wordcountmessage}{
    \begin{center}
        \begin{tabular}{| l | c  c | c  c |}
            \hline
                           & \multicolumn{2}{ c |}{\textbf{Counter}}                     & \multicolumn{2}{ c |}{\textbf{Words}}                 \\
                           & 1 Col & 2 Col                                               & 1 Col & 2 Col                                         \\
            \hline
            Words          & \multicolumn{2}{ c |}{------------}                         & \multicolumn{2}{ c |}{\wordcount}                     \\
            Figure         & \showcounter{figOneCol}      & \showcounter{figTwoCol}      & \numberforafigOneCol      & \numberforafigTwoCol      \\
            Table          & \showcounter{tableOneCol}    & \showcounter{tableTwoCol}    & \numberforatableOneCol    & \numberforatableTwoCol    \\
            Table Row      & \showcounter{tableOneColRow} & \showcounter{tableTwoColRow} & \numberforatableOneColRow & \numberforatableTwoColRow \\
            Eq Row         & \showcounter{eqOneColRow}    & \showcounter{eqTwoColRow}    & \numberforaeqOneColRow    & \numberforaeqTwoColRow    \\
            \hline
            \textbf{Pages} & \multicolumn{2}{ c |}{------------}                         & \multicolumn{2}{ c |}{\textbf{\totalpagecount}}       \\
            \textbf{Total} & \multicolumn{2}{ c |}{------------}                         & \multicolumn{2}{ c |}{\textbf{\totalwordcount}}       \\
            \hline
            \textbf{Remain} & \multicolumn{2}{ c |}{------------}                        & \multicolumn{2}{ c |}{\textbf{\remainwordcount}}       \\
            \hline
        \end{tabular}
    \end{center}
}

% END =================================================================================================================% Word & page count for document

\graphicspath{{../pictures/}}% Path for pictures

%% Apr 2021: AIP requests that the corresponding 
%% email to be moved after the affiliations
\makeatletter
\def\@email#1#2{
	\endgroup
	\patchcmd{\titleblock@produce}
		{\frontmatter@RRAPformat}
		{\frontmatter@RRAPformat{\produce@RRAP{*#1\href{mailto:#2}{#2}}}\frontmatter@RRAPformat}
		{}{}
}
\makeatother

%% Include graphic for one column with specific place 
\newcommand{\includegraphicsOneCol}[3]{
	\begin{center}
		\centering
		\captionsetup{type=figure}
		\includegraphics[width=0.87\linewidth]{#1}
		\captionof{figure}{#2}
		\label{#3}
	\end{center}
  	\increasecounter{fig}{1}
}

%\newcommand{\includegraphicsOneCol}[3]{
%	\begin{figure}[hbt!]
%		\centering
%		\includegraphics[width=0.9\linewidth]{#1}
%		\caption{#2}
%		\label{#3}
%  	\end{figure}
%  	\increasecounter{fig}{1}
%}

%% Include graphic for two column with specific place, figrue* places graphic anywhere...
\newcommand{\includegraphicsTwoCol}[4]{
	\onecolumngrid
	\begin{center}
		\centering
		\captionsetup{type=figure}
    	\includegraphics[width=#4\textwidth]{#1}
		\captionof{figure}{#2}
    	\label{#3}
	\end{center}
	\twocolumngrid
	\increasecounter{fig}{2}
}

%\newcommand{\includegraphicsTwoCol}[3]{
%	\begin{figure*}
%    	\includegraphics[width=0.9\textwidth]{#1}
%		\caption{#2}
%		\label{#3}
%	\end{figure*}
%	\increasecounter{fig}{2}
%}

\newcommand{\words}{
	\immediate\write18{texcount -sum -1 \jobname.tex > wordcount.txt}
}


\newcommand{\wexb}{\omega_{\mathrm{\:E \times B}}}
\newcommand{\hatwexb}{\widehat{\omega}_{\mathrm{\:E \times B}}}
\newcommand{\exb}{\mathrm{\:E}\times\mathrm{B}}
%\newcommand{\hatwexbampvec}{|\hatwexb|_\mathbf{n}}
\newcommand{\hatwexbamp}{|\hatwexb|_{\nzf}}
%\newcommand{\NR}{N_\mathrm{R}}
%\newcommand{\NB}{N_\mathrm{B}}
\newcommand{\NR}{N_\mathrm{R}}
\newcommand{\NB}{N_\mathrm{B}}
\newcommand{\rlt}{R/L_T}
%\newcommand{\rhoth}{\rho_\mathrm{th}}
\newcommand{\rhoth}{\rho}
\newcommand{\vth}{v_{\mathrm{th}}}
\newcommand{\nzf}{n_\mathrm{ZF}}
\newcommand{\kzf}{k_\mathrm{ZF}}
\newcommand{\xcoord}{x}
\newcommand{\ycoord}{y}

% MAIN =====================================================================================================================

\begin{document}

%% TITLE, INFO =============================================================================================================

\title[Size convergence of the $\exb$ staircase pattern in flux tube simulations of ion temperature gradient driven turbulence]
{Size convergence of the $\exb$ staircase pattern in flux tube simulations of ion temperature gradient driven turbulence}

\author{M. Lippert}
	\altaffiliation{Repository of this work: \\ 
					https://github.com/ManeLippert/Bachelorthesis-Shearingrate-Convergence}
\author{F. Rath}
	\altaffiliation{Author to whom correspondence should be addressed: \\
					Florian.Rath@uni-bayreuth.de}
\author{A. G. Peeters}
\affiliation{Physics Department, University of Bayreuth, 95440 Bayreuth, Germany}

\date{\today}

%% ABSTRACT ================================================================================================================

\begin{abstract}
    The radial size convergence of the $\exb$ staircase pattern is adressed in local gradient-driven flux tube simulations of ion temperature gradient (ITG) driven turbulence.
    Its is shown that a mesoscale pattern size of \linebreak $\sim 57-76\,\rhoth$ is inherent to ITG driven turbulence with Cyclone Base Case parameters in the local limit. 
    
\end{abstract}

\maketitle

%% TEXT ====================================================================================================================

%%% INTRODUCTION ===========================================================================================================

%\section{Introduction}
%\label{sec:intro}

Ion temperature gradient driven turbulence close to marginal stability exhibits zonal flow pattern formation on mesoscales, so-called $\exb$ staircase structures \cite{Pradalier2010}.
Such pattern formation has been observed in local gradient-driven flux-tube simulations \cite{Peeters2016, Rath2021}, including collisions \cite{Weikl2017}, as well as global gradient-driven \cite{McMillan2009, Villard2013, Seo2022} and global flux-driven \cite{Pradalier2010, Pradalier2015, Wang2020, Kim2022, Kishimoto2023} studies. 
In global studies, spanning a larger fraction of the minor radius, multiple radial repetitions of staircase structures are usually observed, with a typical pattern size of several ten Larmor radii.
% By contrast, the radial domain size of local flux-tube simulations is often restricted to a similar mesoscale.
By contrast, in the aforementioned local studies the radial size of $\exb$ staircase structures is always found to converge to the radial box size of the flux tube domain.
The above observations lead to the question: 
\textit{Does the basic pattern size always converges to the box size, or is there a typical mesoscale size inherent to staircase structures also in a local flux-tube description?}
The latter case would imply that it is not necessarily global physics, i.e., profile effects, that set (i) the radial size of the $\exb$ staircase pattern and (ii) the scale of avalanche-like transport events. These transport events are usually restricted to $\exb$ staircase structures and considered as a nonlocal transport mechanism \cite{Pradalier2010}. 
In this brief communication the above question is addressed through a box size convergence scan of the same cases close to the nonlinear threshold for turbulence generation as studied in Ref.~\onlinecite{Peeters2016}.\bigskip

%%% SIMULATION SET-UP ======================================================================================================

The gyrokinetic simulations are performed with the non-linear flux tube version of Gyrokinetic Workshop (GKW) \cite{Peeters2009} with adiabatic electron approximation.
In agreement with Ref.~\onlinecite{Peeters2016}, Cyclone Base Case (CBC) like parameters are chosen with an inverse background temperature gradient length $\rlt = 6.0$ and circular concentric flux surfaces. 
The numerical resolution is compliant to the "Standard resolution with 6th order (S6)" set-up of the aforementioned reference, with a somewhat lowered number of parallel velocity grid points.
It has been carefully verified that this modification preserves the same physical outcome as the original study.
A summary of the numerical parameters is given in Tab.~\ref{tab:resolution} and for more details about the definition of individual quantities the reader is referred to Ref.~\onlinecite{Peeters2009, Peeters2016}.
\begin{table}[ht]
	\begin{ruledtabular}
		\begin{tabular}{l | ccccc | ccccc | c | cc}
			& $N_m$ & $N_x$ & $N_s$ & $N_{\nu_\parallel}$ & $N_\mu$ & $D$ & $\nu_d$           & $D_{\nu_\parallel}$ & $D_x$ & $D_y$ & Order & $k_y\rho$ & $k_x\rho$ \\
			\hline
			S6   & 21    & 83    & 16    & 48                  & 9       & 1   & $|\nu_\parallel|$ & 0.2                 & 0.1   & 0.1   & 6     & 1.4       & 2.1       \\
		\end{tabular}
	\end{ruledtabular}
	\caption{
		Resolution used in this paper for further information the author links to Ref.~\onlinecite{Peeters2016}. %: Number of toroidal modes $N_m$, number of radial modes $N_x$, number of grid points along the magnetic field $N_s$,number of parallel velocity grid points $N_{\nu_\parallel}$, number of magnetic moment grid points $N_\mu$, dissipation coefficient used in convection along the magnetic field $D$,the velocity in the dissipation scheme $\nu_d$, dissipation coefficient used in the trapping term $D_{\nu_\parallel}$, damping coefficient of radial modes $D_x$, damping coefficient of toroidal modes $D_y$, order of the scheme used for the zonal mode, maximum poloidal wave vector $k_y\rho$, and maximum radial wave vector $k_x\rho$
	}
	\label{tab:resolution}
\end{table}

%%% DIAGNOSTICS ================================================================================================================

In the following the box size is increased relative to the standard box size $(L_\xcoord,~L_\ycoord) = (76,~89)\,\rhoth$ in the radial and binormal direction. Here, $\xcoord$ is the radial coordinate that labels the flux surfaces normalized by the thermal Larmor radius $\rhoth$, $\ycoord$ labels the field lines and is an approximate binormal coordinate. Together with the coordinate $s$ which parameterizes the length along the field lines and is referred to as the parallel coordinate these quantities form the Hamada coordinates \cite{Hamada1958}.
The increased box sizes are indicated by the real parameter $\NR$ for radial and $\NB$ for the binormal direction with the nomenclature $\NR\times \NB$ throughout this work.
Note that, the number of modes in the respective direction, i.e., $N_x$ and $N_m$, respectively, is always adapted accordingly to retain a spatial resolution compliant to the standard resolution [Tab.~\ref{tab:resolution}] and standard box size. \\
The $\exb$ staircase pattern is manifest as radial structure formation in the $\exb$ shearing rate defined by\cite{Rath2016, Pueschel2008, Rath2016, Peeters2016}
\begin{equation}
	\wexb = \frac{1}{2} \frac{\partial^2 \langle \phi \rangle}{\partial \xcoord^2},
	\label{eq:shearingrate}
\end{equation}
where $\langle \phi \rangle$ is the zonal electrostatic potential normalized by $\rho_\ast T/e$ ($\rho_\ast = \rhoth/R$ is the thermal Larmor radius normalized with the major radius $R$, $T$ is the temperature, $e$ is the elementary charge).
The zonal potential is calculated from the electrostatic potential $\phi$ on the two-dimensional $\xcoord$-$\ycoord$-plane at the low field side according to\cite{Rath2021}
\begin{equation}
\langle \phi \rangle = \frac{1}{L_\ycoord} \int_0^{L_\ycoord} \mathrm{d}\ycoord ~ \phi(\xcoord,\ycoord,s=0).
\end{equation}
The E$\times$B shearing rate $\wexb$ is the radial derivative of the advecting zonal flow velocity \cite{Hahm1995, Waltz1998} and quantifies the zonal flow induced shearing of turbulent structures \cite{Biglari1990, Hahm1995, Burnell1997}. \\
Consistent with Ref.~\onlinecite{Peeters2016} the turbulence level is quantified by the turbulent heat conduction coefficient $\chi$, which is normalized by $\rhoth^2 \vth/R$ ($\vth = \sqrt{2 T/m}$ is the thermal velocity and $m$ is the mass). Furthermore, quantities $\rhoth$, $R$, $T$, $\vth$ and $m$ are referenced quantities from Ref. \onlinecite{Peeters2016,Peeters2009}.
\newpage
In order to diagnose the temporal evolution of the staircase pattern and to obtain an estimate of its amplitude the radial Fourier transform of the $\exb$ shearing rate is considered. 
It is defined by
\begin{equation}
	\wexb = \sum_{\kzf} \hatwexb(\kzf,t) \, \exp(\mathrm{i} \kzf \xcoord),
	\label{eq:shearingrate_fourier}
\end{equation}
where $\hatwexb$ is the complex Fourier coefficient and \linebreak $\kzf = 2\pi \nzf/L_\xcoord$
% explicitly introduce the terms zonal flow wave vector and zonal flow mode number for later usage
defines the zonal flow wave vector with the zonal flow mode number $\nzf$ ranging in $-(N_\xcoord -1)/2 \leq \nzf \leq (N_\xcoord -1)/2 $.
% define the zonal flow modes' amplitude in terms of the shearing rate
Based on the definitions above, the shear carried by the zonal flow mode with wave vector $\kzf$ is defined by $\hatwexbamp = 2 |\hatwexb(\kzf,t)|$. 
% introduces the term basic mode of the pattern for later usage
In general, the zonal flow mode that dominates the $\exb$ staircase pattern, also referred to as the \textit{basic mode} of the pattern in this work, exhibits the maximum amplitude in the spectrum $\hatwexbamp$.\\


%%% RESULTS ================================================================================================================

%\section{Results}
%\label{sec:results}

% what kind of analysis
In the first test the radial box size is increased while the binormal box size is kept fixed to the standard size. The scan covers the realizations $\NR\times\NB \in [ 1\times1,~2\times1,~3\times1,~4\times1]$.
% short introduction of the turbulence - zonal flow dynamics; cite peeters2016 explicitly to avoid any discussions lateron
Each realization exhibits an initial quasi-stationary turbulent phase and a second final \cite{Peeters2016} phase with almost suppressed turbulence [Fig.~\ref{fig:wexb-eflux-1-2-3-4x1-comparison}\,(a)].
% introduce the fully developed staircase structure
The latter state is indicative for the presence of a fully developed staircase pattern as depicted in Fig.~\ref{fig:wexb-stable-comparison}. 
This type of structure is characterized by intervals of almost constant shear with alternating sign satisfying the Waltz criterion $|\wexb| \approx \gamma$\cite{Waltz1994, Waltz1998} ($\gamma$ is the growth rate of the most unstable linear ITG driven Eigenmode), connected by steep flanks where $\wexb$ crosses zero.
% insofar the results of peeters2016 are well reproduced; now turn to the description of the main result of this test
Fig.~\ref{fig:wexb-stable-comparison}\,(a) shows a striking repetition of the staircase structure, with the number of repetitions equal to $\NR$.
% yes, the radial size converges; but even stronger result: the standard box size already allows for the correct staircase size
Hence, the basic size of the pattern not only converges with increasing radial box size, the converged radial size turns out to at least roughly agree with the standard radial box size of Ref.~\onlinecite{Peeters2016}. \\
% cautios words
Due to the lack of a substantial turbulent drive in the final suppressed state no further zonal flow evolution is observed [Fig.~\ref{fig:wexb-eflux-1-2-3-4x1-comparison}\,(b)] and one might critically ask whether the structures shown in Fig.~\ref{fig:wexb-stable-comparison} represent the real converged pattern in a statistical sense. 
% very long initial quasi-stationary states
Note that in the $3 \times 1$ case the initial quasi-stationary turbulent state extends up to a few $\sim 10^4\,R/\vth$.
% the \nzf=3 zonal flow mode does show a temporal evolution with a few cycles covered by the initial turbulent phase -> this allows for some degree of statistical argument
During this period the zonal flow mode with $\nzf = 3$, i.e., the mode that dominates the staircase pattern in final suppressed phase, undergoes a long-term evolution with a typical time scale of several $\sim 10^3\,R/\vth$. 
Hence, several of such cycles are covered by the initial turbulent phase, which is evident from the occurrence of phases with reduced amplitude around $t \approx 8000\,R/\vth$ and $t \approx 18000\,    R/\vth$.
% first important point: competition between the \nzf = 3 and \nzf = 4 mode -> the \nzf = 4 mode is of shorter length scale compared to the \nzf = 3 mode, hence, no convergence to the box scale
It is the $\nzf = 4$ zonal flow mode, i.e., the next shorter radial scale mode, that dominates the shear spectrum $\hatwexbamp$ in the latter two phases (not shown). This demonstrates a competition between the $\nzf = 3$ and $\nzf = 4$ modes.
% second important point: no secular growth of the box scale zonal mode
Most importantly, no secular growth of the $\nzf = 1$ (box scale) zonal flow mode is observed during the entire quasi-stationary turbulent phase [Fig.~\ref{fig:wexb-eflux-1-2-3-4x1-comparison}\,(b) dotted line].
% summary
The above discussion indicates that although the $\nzf = 3,~4$ zonal modes compete, the pattern scale does not converge to the radial box scale but rather to a mesoscale of $\sim 57 - 76\,\rhoth$ (i.e., $\nzf = 4,~3$ in the $3\times1$ case). \\

% what kind of analysis
Since the radially elongated simulation domain might inhibit the development of isotropic turbulent structures, in the second test the radial and binormal box size is increased simultaneously.
This scan covers the realizations $\NR\times\NB \in [1\times1,~1.5\times1.5,~2\times2,~2.5\times2.5,~3\times3]$.
% mention shorter times scales for stabilization
Interestingly, suppression of the turbulence by the emergence of a fully developed staircase pattern almost always occurs after \linebreak $\sim 1000~R/\vth$ [Fig. \ref{fig:eflux-1x1-2x2-3x3-comparison}], i.e., significantly faster compared to the $3\times1$ and $4\times1$ realizations. 
% Although the computational costs per time step is higher, the difference in the time scales renders this test computationally cheaper.
%
% the main result of this test: confirmation of wave length convergence
As shown in Fig.~\ref{fig:wexb-stable-comparison}\,(b) also this test confirms the convergence of the staircase pattern size to a typical mesoscale that is distinct from the radial box size in the $\NR > 1$ realizations.

\includegraphicsTwoCol{{Briefcommunication/1}.pdf}{
	\textbf{(a)} Time traces of the heat conduction coefficient $\chi$ for $\rlt = 6.0$ for radial increased box sizes\\
	\textbf{(b)} Time traces of $\hatwexbamp$ for radial increased box sizes
}{fig:wexb-eflux-1-2-3-4x1-comparison}{}

\includegraphicsTwoCol{{Briefcommunication/2}.pdf}{
	Comparison of shearing rate $\wexb$ for each box sizes scan averaged over given time interval and the growth rate $\pm \gamma$ of the most unstable linear ITG driven Eigenmode. The staircase structures are radially shifted with respect to each over till alignment for better visibility.\\
	\begin{tabular}{l l l l l l l l l}
		\textbf{(a) radial:}    & $t_{1\times 1}$   & $\in [2000, 5000]$,   & $t_{2\times 1}$     & $\in [15000, 18000]$, & $t_{3\times 1}$ & $\in [43000, 45000]$, & $t_{4\times 1}$     & $\in [26000, 28000]$ \\
		\textbf{(b) isotropic:} & $t_{1\times 1}$   & $\in [2000, 5000]$,   & $t_{1.5\times 1.5}$ & $\in [7000, 8000]$,   & $t_{2\times 2}$ & $\in [2000, 3000]$,   & $t_{2.5\times 2.5}$ & $\in [2000, 3000]$   \\
		                        & $t_{3\times 3}$   & $\in [2000, 3000]$    &                     &                       &                 &                       &                     &                      \\
		\textbf{(c) binormal:}  & $t_{3\times 1.5}$ & $\in [2000, 3000]$,   & $t_{3\times 2.5}$   & $\in [2000, 3000]$,   & $t_{3\times 3}$ & $\in [2000, 3000]$,   & $t_{3\times 5}$     & $\in [1000, 3000]$   \\
	\end{tabular}
}{fig:wexb-stable-comparison}{0.98}

\includegraphicsOneCol{{Briefcommunication/3}.pdf}{
	Time traces of the heat conduction coefficient $\chi$ for $\rlt = 6.0$ for isotropic increased box sizes
}{fig:eflux-1x1-2x2-3x3-comparison}

% discuss sumewhat smaller radial size
By contrast to the radial box size scan the $3\times3$ realization shows a stationary pattern with four repetitions of the fully developed staircase structure, i.e., a somewhat smaller pattern size. 
Whether this is related to a possible pattern size dependence on the binormal box size or to the competition between patterns with the two sizes $\lambda \in [57,~ 76]\,\rhoth$ as observed in the first test is addressed in the next paragraph.
\newpage
% what kind of analysis
In a third test the binormal box size is varied with the radial box size fixed to $\NR = 3$.
This test covers the realizations $\NR \times \NB \in [3\times1.5,~3\times2.5,~3\times3,~3\times5]$. As in the isotropic scan the turbulence subdued and a fully developed staircase pattern forms after $\sim 2000\,R/\vth$ [Fig.~\ref{fig:eflux-3x1.5-2.5-3-5-comparison}]. The convergence of staircase pattern can be seen in Fig.~\ref{fig:wexb-stable-comparison}\,(c) and confirms again a size of a typical mesoscale. Fig.~\ref{fig:wexb-stable-comparison}\,(c) also confirms that indeed a competition between patterns with two sizes $\lambda \in [57,~ 76]\,\rhoth$ causing the different results for $3 \times 1$ and $3\times 3$. The zonal flow mode number varies between $\nzf = 3,4$ which can be seen in Fig.~\ref{fig:wexb-stable-comparison}\,(c) in the $3\times 2.5$ realization. The staircase structure has a pattern between $3$ and $4$ repetitions which get represented in the second repetition with no signifciant plateau at positive shear. Instead the pattern returns immediately after reaching the maximum shear ($+ \gamma$) to the minimum shear ($- \gamma$) of the third repetition in a steep flank. The Fourier analysis of this case yields no definitely basic mode rather two dominating modes with $\nzf = 3, 4$ with a fraction of the maximum amplitude $\hatwexbamp$ each (not shown).

\includegraphicsOneCol{{Briefcommunication/4}.pdf}{
	Time traces of the heat conduction coefficient $\chi$ for $\rlt = 6.0$ for binormal increased box sizes
}{fig:eflux-3x1.5-2.5-3-5-comparison}\bigskip

% what kind of test
In the final test the inverse background temperature gradient length $\rlt$ is varied at fixed $3\times3$ box size.
Since suppression of turbulence usually occurs at later times when approaching the finite heat flux threshold from below \cite{Peeters2016}, the analysis aims to lengthen the phase during which the zonal flow varies in time due to turbulent Reynolds stresses.
This scan covers realizations with $\rlt \in [6.0,~6.2,~6.4]$.
% what is the finite heat flux threshold
In the case of $\rlt = 6.2$ turbulence suppression is observed for $t > 11000\,R/\vth$, while stationary turbulence during the entire simulation time trace of $12000\,R/\vth$ is found for $\rlt = 6.4$.
The finite heat flux threshold, hence, is $\rlt|_\mathrm{finite} = 6.3 \pm 0.1$ in accordance to Ref. \onlinecite{Peeters2016}.
% Which stationay radial size occurs????
Although the initial quasi-stationary turbulence in the former case is significantly longer compared to the $\rlt = 6.2$ realization discussed in the second test, a stationary pattern with basic zonal flow mode $\nzf = 3$ establishes. 
% Does it confirm our previous tests
Again, the $\nzf = 1$ (box scale) zonal flow mode does not grow secularly during the entire turbulent phase.
Also, this test confirms the statistical soundness of the converged pattern size of $\sim 57 - 76\,\rhoth$.\\

\bigskip
\bigskip
\bigskip

%%% CONCLUSION =============================================================================================================

%\section{Conclusion}
%\label{sec:conclusion}

% what has been done
Through careful tests this brief communication confirms the radial size convergence of the $\exb$ staircase pattern in local gyrokinetic flux tube simulations of ion temperature gradient (ITG) driven turbulence.
% mention the scale explicitly, and also that it is the scale compliant to CBC parameters (for other plasma parameters it might differ)
A mesoscale pattern size of $\sim 57 - 76\,\rhoth$ is found to be intrinsic to ITG driven turbulence for Cyclone Base Case parameters.
% compare to other studies
This length scale is somewhat larger compared to results from global studies with finite $\rho_\ast$, which report of a few $10\,\rhoth$ \cite{Pradalier2010}, and has to be considered the proper mesoscale in the local limit $\rho_\ast \rightarrow 0$.  
% relevance of the outcome
The occurrence of this mesoscale implies that non-locality, in terms of Ref. \onlinecite{Pradalier2010}, is inherent to ITG driven turbulence, since avalanches are spatially organized by the $\exb$ staircase pattern \cite{McMillan2009, Pradalier2010, Rath2016, Peeters2016}. 

%% APPENDIX ================================================================================================================

\section*{Data Availability}
The data that support the findings of this study are available from the corresponding author upon reasonable request. 

% WORDCOUNT ================================================================================================================

%\wordcountmessage

% BIBLIOGRAPHY =============================================================================================================

\nocite{}
\bibliography{references.bib}% Produces the bibliography via BibTeX.

\end{document}
% END MAIN =================================================================================================================
% ****** End of file aipsamp.tex ******
