% AUTHOR ===================================================================================================================
% Manuel Lippert (GitHub: ManeLippert (https://github.com/ManeLippert))
% ==========================================================================================================================

% PREAMBLE =================================================================================================================

% ****** Start of file aipsamp.tex ******
%
%   This file is part of the AIP files in the AIP distribution for REVTeX 4.
%   Version 4.1 of REVTeX, October 2009
%
%   Copyright (c) 2009 American Institute of Physics.
%
%   See the AIP README file for restrictions and more information.
%
% TeX'ing this file requires that you have AMS-LaTeX 2.0 installed
% as well as the rest of the prerequisites for REVTeX 4.1
% 
% It also requires running BibTeX. The commands are as follows:
%
%  1)  latex  aipsamp
%  2)  bibtex aipsamp
%  3)  latex  aipsamp
%  4)  latex  aipsamp
%
% Use this file as a source of example code for your aip document.
% Use the file aiptemplate.tex as a template for your document.

% DOCUMENT =================================================================================================================

\documentclass[aip, amsmath, amssymb, reprint, twocolumn]{revtex4-1}

% documentclass revtex4-1 options:
% - aip,
% - jmp,
% - bmf,
% - sd,
% - rsi,
% - amsmath,amssymb,
% - preprint,%
% - reprint,%
% - author-year,%
% - author-numerical,%
% - Conference Proceedings

\preprint{AIP/123-QED}

% PACKAGES =================================================================================================================

\usepackage[utf8]{inputenc}
\usepackage[T1]{fontenc}
\usepackage{mathptmx}
\usepackage{etoolbox}
\usepackage{graphicx}% Include figure files
\usepackage{dcolumn}% Align table columns on decimal point
\usepackage{bm}% bold math
%\usepackage[mathlines]{lineno}% Enable numbering of text and display math
%\linenumbers\relax % Commence numbering lines

%% Additional
\usepackage{caption}
\usepackage{hyperref}

% FUNCTIONS, INPUT =========================================================================================================

% AUTHOR ==============================================================================================================
% Manuel Lippert (GitHub: ManeLippert (https://github.com/ManeLippert))
% =====================================================================================================================

\usepackage{xparse}
\usepackage{booktabs}
\usepackage{ifthen}
\usepackage{xcolor}
\usepackage{refcount}

% COUNTER, WORD COUNT, VARIABLE =======================================================================================

\def\wordlimit{3500}
\def\pagelimit{4}

\newcounter{totalwordcounter}
\newcounter{remainwordcounter}
\addtocounter{remainwordcounter}{\wordlimit}
\newcounter{wordcounter}

% Figures
\newcounter{figOneCol}      \newcommand{\numberforafigOneCol}{200}
\newcounter{figTwoCol}      \newcommand{\numberforafigTwoCol}{400}

% Tables
\newcounter{tableOneCol}    \newcommand{\numberforatableOneCol}{13}
\newcounter{tableOneColRow} \newcommand{\numberforatableOneColRow}{5}
\newcounter{tableTwoCol}    \newcommand{\numberforatableTwoCol}{26}
\newcounter{tableTwoColRow} \newcommand{\numberforatableTwoColRow}{13}

% Equations
\newcounter{eqOneColRow}    \newcommand{\numberforaeqOneColRow}{7}
\newcounter{eqTwoColRow}    \newcommand{\numberforaeqTwoColRow}{13}

% FUNCTIONS ===========================================================================================================

% Count words in document
\newread\somefile
\makeatletter
\NewDocumentCommand{\wordcount}{s}{%
  \immediate\write18{texcount -sum -1 \jobname.tex > wordcount.txt}%
  \immediate\openin\somefile=wordcount.txt%
  \read\somefile to \@@localdummy%
  \immediate\closein\somefile%
  \setcounter{wordcounter}{\@@localdummy}%
  \IfBooleanF{#1}{%
  \@@localdummy%   print only if not starred version
  }%
}
\makeatother

% Total word count and colors output
\newcommand{\totalwordcount}{
    % Words
    \addtocounter{totalwordcounter}{\value{wordcounter}}
    % Figures
    \addtocounter{totalwordcounter}{\number\numexpr\numberforafigOneCol*\value{figOneCol}}
    \addtocounter{totalwordcounter}{\number\numexpr\numberforafigTwoCol*\value{figTwoCol}}
    % Tables
    \addtocounter{totalwordcounter}{\number\numexpr\numberforatableOneColRow*\value{tableOneColRow}}
    \addtocounter{totalwordcounter}{\number\numexpr\numberforatableOneCol*\value{tableOneCol}}
    \addtocounter{totalwordcounter}{\number\numexpr\numberforatableTwoColRow*\value{tableTwoColRow}}
    \addtocounter{totalwordcounter}{\number\numexpr\numberforatableTwoCol*\value{tableTwoCol}}
    % Equations
    \addtocounter{totalwordcounter}{\number\numexpr\numberforaeqOneColRow*\value{eqOneColRow}}
    \addtocounter{totalwordcounter}{\number\numexpr\numberforaeqTwoColRow*\value{eqTwoColRow}}

    %\number\value{totalwordcounter}
    %\renewcommand{\totalwordcount}{\number\value{totalwordcounter}}

    % Color Output
    \ifnum\value{totalwordcounter}>\wordlimit
        \textcolor{red}{\showcounter{totalwordcounter}}
    \else
        \textcolor{teal}{\showcounter{totalwordcounter}}
    \fi
}

% Display value of counter
\newcommand{\showcounter}[1]{\number\value{#1}}

% Subtract counters
\newcommand{\subtocounter}[2]{\setcounter{#1}{\numexpr\value{#1}-\value{#2}}}

\newcommand{\increasecounter}[3]{
    \ifthenelse{\equal{#2}{1}}{\def\col{OneCol}}{\def\col{TwoCol}}

    \ifthenelse{\equal{#1}{fig}}{\addtocounter{#1\col}{1}}
        {\ifthenelse{\equal{#1}{eq}}{\addtocounter{#1\col Row}{#3}}
            {\addtocounter{#1\col}{1} \addtocounter{#1\col Row}{#3}}
        }
}

% Color output for page numbers
\newcommand{\totalpagecount}{
    \ifnum\getpagerefnumber{LastPage}>\pagelimit
        \textcolor{red}{\pageref*{LastPage}}
    \else
        \textcolor{teal}{\pageref*{LastPage}}
    \fi
}

% Color output for remain counter
\newcommand{\remainwordcount}{

    \subtocounter{remainwordcounter}{totalwordcounter}

    \ifnum\value{remainwordcounter}<0
        \textcolor{red}{\showcounter{remainwordcounter}}
    \else
        \textcolor{teal}{\showcounter{remainwordcounter}}
    \fi
}

% Display word count message
\newcommand{\wordcountmessage}{
    \begin{center}
        \begin{tabular}{| l | c  c | c  c |}
            \hline
                           & \multicolumn{2}{ c |}{\textbf{Counter}}                     & \multicolumn{2}{ c |}{\textbf{Words}}                 \\
                           & 1 Col & 2 Col                                               & 1 Col & 2 Col                                         \\
            \hline
            Words          & \multicolumn{2}{ c |}{------------}                         & \multicolumn{2}{ c |}{\wordcount}                     \\
            Figure         & \showcounter{figOneCol}      & \showcounter{figTwoCol}      & \numberforafigOneCol      & \numberforafigTwoCol      \\
            Table          & \showcounter{tableOneCol}    & \showcounter{tableTwoCol}    & \numberforatableOneCol    & \numberforatableTwoCol    \\
            Table Row      & \showcounter{tableOneColRow} & \showcounter{tableTwoColRow} & \numberforatableOneColRow & \numberforatableTwoColRow \\
            Eq Row         & \showcounter{eqOneColRow}    & \showcounter{eqTwoColRow}    & \numberforaeqOneColRow    & \numberforaeqTwoColRow    \\
            \hline
            \textbf{Pages} & \multicolumn{2}{ c |}{------------}                         & \multicolumn{2}{ c |}{\textbf{\totalpagecount}}       \\
            \textbf{Total} & \multicolumn{2}{ c |}{------------}                         & \multicolumn{2}{ c |}{\textbf{\totalwordcount}}       \\
            \hline
            \textbf{Remain} & \multicolumn{2}{ c |}{------------}                        & \multicolumn{2}{ c |}{\textbf{\remainwordcount}}       \\
            \hline
        \end{tabular}
    \end{center}
}

% END =================================================================================================================% Word & page count for document

\graphicspath{{../pictures/}}% Path for pictures

%% Apr 2021: AIP requests that the corresponding 
%% email to be moved after the affiliations
\makeatletter
\def\@email#1#2{
	\endgroup
	\patchcmd{\titleblock@produce}
		{\frontmatter@RRAPformat}
		{\frontmatter@RRAPformat{\produce@RRAP{*#1\href{mailto:#2}{#2}}}\frontmatter@RRAPformat}
		{}{}
}
\makeatother

%% Include graphic for one column with specific place 
\newcommand{\includegraphicsOneCol}[3]{
	\begin{figure}[h]
		\includegraphics[width=0.9\linewidth]{#1}
		\caption{#2}
  	\end{figure}
	\label{#3}
  	\increasecounter{fig}{1}
}

%% Include graphic for two column with specific place, figrue* places graphic anywhere...
\newcommand{\includegraphicsTwoCol}[3]{
	\onecolumngrid
	\begin{center}
		\captionsetup{type=figure}
    	\includegraphics[width=\textwidth]{#1}
		\captionof{figure}{#2}
	\end{center}
	\twocolumngrid
	\label{#3}
	\increasecounter{fig}{2}
}

% MAIN =====================================================================================================================

\begin{document}

%% TITLE, INFO =============================================================================================================

%\title[Convergence of shearing rate $\omega_{\mathrm{E}\times\mathrm{B}}$ with boxsize]{
%	Convergence of shearing rate $\omega_{\mathrm{E}\times\mathrm{B}}$ with boxsize
%}

%\author{M. Lippert}
%	\altaffiliation{GitHub Repository for files reagarding this paper: \\ 
%					https://github.com/ManeLippert/Bachelorthesis-ZonalFlows}
%\author{F. Rath}
%	\altaffiliation{Author to whom correspondence should be addressed: \\ 
%					Florian.Rath1@uni-bayreuth.de}
%	\email{Florian.Rath1@uni-bayreuth.de}
%\author{A. G. Peeters}
%\affiliation{Physics Department, University of Bayreuth, 95440 Bayreuth, Germany}

%\date{\today}

%% ABSTRACT ================================================================================================================

%\begin{abstract}
%	An article usually includes an abstract, a concise summary of the work
%	covered at length in the main body of the article. It is used for
%	secondary publications and for information retrieval purposes. 
%\end{abstract}

%\maketitle

%% TEXT ====================================================================================================================

%%% INTRODUCTION ===========================================================================================================

It is known that radially sheared zonal flows plays a significant role in nonlinear stabilization in tokamak plasmas. \cite{WACooper1988,doi:10.1063/1.859529,doi:10.1063/1.873896}. 
Through advection on the sheared zonal flows the turbulent structure in plasma gets deformed and tilted that causes an $E \times B$ nonlinearty. \cite{doi:10.1063/1.859529, doi:10.1063/1.871313, doi:10.1063/1.872367}
Zonal flows mediate spectral energy transfer to larger radial wave vectors. \cite{doi:10.1063/1.3033206, doi:10.1063/1.3675855, PhysRevLett.120.175002}
The strength of the shearing process is the $E \times B$ shearing rate $\omega_{\mathrm{E \times B}}$ which is the radial derivative of the advecting zonal flow velocity. \cite{doi:10.1063/1.871313, doi:10.1063/1.872847}
The shearing rate $\omega_{\mathrm{E \times B}}$ is defined as 
\begin{equation}
	\omega_{\mathrm{E \times B}} = \frac{1}{2} \frac{\partial^2 \langle \Phi \rangle}{\partial \psi^2}
	\label{eq:shearingrate}
\end{equation}
where $\langle \Phi \rangle$ is the zonal electrostatic potential and $\psi$ the radial coordinate that labels the flux surfaces. \cite{doi:10.1063/1.4952621, doi:10.1063/1.5035184, doi:10.1063/1.3005380}
It was shown that the nonlinear threshold for turbulence is directly related to shear stabilization. \cite{doi:10.1063/1.873896} 
Often the shear stabilization is expressed in the empirical Waltz rule $\omega_{\mathrm{E \times B}} \sim \gamma$, \cite{doi:10.1063/1.870934, doi:10.1063/1.872847} where $\gamma$ is defined as the maximum linear growth rate in the unstable mode. In the discovered zonal flows, also known as $E \times B$ staircase \cite{PhysRevE.82.025401}, exhibit amplitudes in terms of the $E \times B$ shearing rate satisfying the stabilization criteria. \cite{doi:10.1063/1.4952621, doi:10.1063/1.4961231}

\includegraphicsOneCol{Comparison/Boxsize/S6_rlt6.0_boxsize1x1-2x2-3x3_Ns16_Nvpar48_Nmu9_eflux_comparison.pdf}{Test}{fig:eflux}

%\includegraphicsOneCol{Comparison/Gradient-Length/S6_rlt6.0-6.3_boxsize1x1_Ns16_Nvpar64_Nmu9_eflux_comparison.pdf}{Test}{fig:eflux}

\includegraphicsOneCol{S6_rlt6.0/boxsize3x1/Ns16/Nvpar48/Nmu9/S6_rlt6.0_boxsize3x1_Ns16_Nvpar48_Nmu9_wexb_selection.pdf}{Test}{fig:eflux}

\includegraphicsOneCol{S6_rlt6.0/boxsize3x3/Ns16/Nvpar48/Nmu9/S6_rlt6.0_boxsize3x3_Ns16_Nvpar48_Nmu9_wexb_selection.pdf}{Test}{fig:eflux}

%\includegraphicsOneCol{Comparison/Gradient-Length/S6_rlt6.0-6.2_boxsize2x2_Ns16_Nvpar48-64_Nmu9_eflux_comparison.pdf}{Test}{fig:eflux}

\includegraphicsTwoCol{Comparison/Boxsize/S6_rlt6.0_boxsize2x1-2-3x1-3_Ns16_Nvpar48_Nmu9_eflux_comparison.pdf}{Test}{fig:test}

\includegraphicsTwoCol{Comparison/Boxsize/S6_rlt6.0_boxsize1-2-3-4x1_Ns16_Nvpar48_Nmu9_wexb_comparison.pdf}{Test}{fig:test}

%\includegraphicsTwoCol{Comparison/Boxsize/S6_rlt6.0_boxsize1-2-3-4x1_Ns16_Nvpar48_Nmu9_eflux_comparison.pdf}{Test}{fig:test}

\includegraphicsTwoCol{Comparison/Boxsize/S6_rlt6.0_boxsize1-2-3-4x1_Ns16_Nvpar48_Nmu9_comparison.pdf}{Test}{fig:test}

\increasecounter{fig}{2}

\section*{Data Availability}
The data that support the findings of this study are available from the corresponding author upon reasonable request.

% WORDCOUNT ================================================================================================================

\wordcountmessage

% BIBLIOGRAPHY =============================================================================================================

%\nocite{*}
\bibliography{references.bib}% Produces the bibliography via BibTeX.

\end{document}
% END MAIN =================================================================================================================
% ****** End of file aipsamp.tex ******