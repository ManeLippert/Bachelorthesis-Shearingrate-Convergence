\documentstyle [12pt, epsf] {article}

\topmargin 0.00in
\oddsidemargin 0.25in
\textheight 9.00in
\textwidth 6.00in
\pagestyle{plain}
\renewcommand{\baselinestretch}{1.4}

\begin{document}
\title{
PDP1\\ Plasma Device 1 Dimensional Bounded Electrostatic Code
}
\author{
Plasma Theory and Simulation Group\\
EECS Department,
University of California,\\ Berkeley, CA 94720}

\date{}

%
\begin{figure}
\begin{center}
\leavevmode
\hbox{%
\epsfysize 4.0in
\epsfxsize 4.0in
\epsffile{pdp1.epsi}}
\end{center}
\end{figure}
%
\maketitle
\newpage
\tableofcontents
\newpage

\begin{section}
{\bf INTRODUCTION}

XPDP1 is a bounded electrostatic code for simulating 1 dimensional Plasma
Devices running on Unix workstations with X-Windows, and PC's with an X-Windows
emulator.  The code simulates a bounded plasma plus external circuit system
whose characteristics, including particles and electrostatic fields, are
specified by the user at run time using an input file.  The external circuit
includes R, L, and C elements, as well as AC, DC, and ramped current and
voltage sources.  The simulation proceeds in real-time, with the user viewing
output as the code runs in the form of various user specified diagnostics which
are updated each time step (animation).
\vspace{0.1in}

\noindent
The code compiles with standard C compilers and requires X-Windows libraries
(X10 or higher).  This version of PDP1 differs from the Cray FORTRAN version
(PDW1 - Plasma Device Workshop) originated by W. S. Lawson on several physics
issues summarized below.  The X-Windows version provides interactive animation
whereas the FORTRAN version produces output files requiring post processing.

\begin{subsection}
{\bf Scope}

   This document describes the XPDP1 programs running on the workstations and
   UNICOS Cray environment.  The general physics issues involved in a bounded
   plasma simulation are discussed briefly.  Program installation, operation,
   and modification are discussed.  In addition, the library of input files
   accompanying XPDP1 is described, and the guidelines to generate new input
   files are provided.
\vspace{.2in}

\noindent
   This manual makes no attempt to explain exhaustively the physics and
   computational issues of particle simulation.  Many comprehensive texts on
   particle simulation are available \cite{kn:bird}\cite{kn:hock}.
   For information on the Cray FORTRAN
   version of PDW1, see W. S. Lawson \cite{kn:law}.
\vspace{.2in}

\noindent
   Some familiarity with plasma physics is required to understand the results
   of the simulations and generate new simulations.  Knowledge of numerical
   analysis and/or particle simulation is useful for modification of the code
   and understanding of numerical errors which can occur in any computer
   simulation.

\end{subsection}

\begin{subsection}
{\bf Objectives}
   The objectives of this package include distribution of the $XPDP1$ codes in
   a
   form which makes it accessible to large and small research facilities,
   universities, and individuals interested in plasma simulation.  The Unix and
   X-Windows environments provide a vehicle for distribution since nearly every
   researcher and student has access to this type of machine.  The trend in
   recent years has been one of decreasing cost and increasing performance and
   capability.
\end{subsection}
		
\begin{subsection}
{\bf History}
   \mbox{PDW1} was originally written by W. S. Lawson for a plasma device seminar on
   axially bounded plasma systems at the University of California at Berkeley
   in 1983. We will discuss
   differences with the Cray \mbox{FORTRAN} version below as appropriate.
\vspace{.2in}

\noindent
   The \mbox{FORTRAN} code was translated into \mbox{Cray-C} by T. L.
   Crystal in 1987.
   Crystal made simplifications in the external circuit as well as a number of
   programming-related changes.  \mbox{PDP1} has significant differences 
   from these
   previous versions of \mbox{PDW1}, containing more accurate and 
   versatile circuit
   solvers, simultaneous solution of the circuit and potential equation coupled
   through boundary conditions at the plasma wall, a basic ionization model,
   MKS units, and Fourier transform diagnostics as well as many additional
   diagnostics and tools.
\end{subsection}

\begin{subsection}
{\bf Enhancements}
   This section summarizes the enhancements made to the \mbox{PC}
   version of \mbox{PDP1}
   since the original version (\mbox{PDW1}) written in \mbox{FORTRAN}
   by Lawson.  The
   enhancements are presented in the order they occurred, referenced by version
   number.  Note that the original version has no version number, and the
   \mbox{Cray-C} version due to Crystal has been given version number 
   \mbox{PDW1-1.0}
   although it had no explicit version number.

\begin{subsubsection}
{\bf Version \mbox{PDW1-2.0}}
      The code was translated into Microsoft C 5.~1 running under
      \mbox{MS-DOS} on \mbox{IBM}
      \mbox{PC} compatible microcomputers.  The diagnostics code was replaced with
      real time graphics animation in \mbox{EGA} resolution (640x350 pixels, 16
      colors).  The code had limitations on particles, diagnostics, etc., and
      the user interface consisted of mnemonic keys to display snapshots of
      diagnostics.  The code could only run while displaying phase space, and
      had a maximum time-step limitation.
\vspace{.2in}

\noindent
      Version \mbox{PDW1-2.0} was never officially released due to the rapid
      development of \mbox{PDP1} described below.
\end{subsubsection}

\begin{subsubsection}
{\bf Version 1.0}

      Version 1.0 is a major overhaul providing PDP1 with an improved user
      interface and output capabilities.  The core of the user interface,
      designed and written by J. P. Verboncoeur and V. Vahedi\cite{kn:verb}, 
      handles the
      programming issues of the simulation.  These include keyboard handling,
      screen graphics, and printer output for Postscript and IBM Graphics
      printers (including the Epson FX compatible family of printers) as well
      as CGM (Computer Graphics Metafile) output for editing by many drawing
      programs.  The code is implemented in an object-oriented style in
      standard C to facilitate enhancements, including new menu items and
      diagnostics.
%
\begin{figure}
\begin{center}
\leavevmode
\hbox{%
\epsfysize 4.0in
\epsfxsize 4.0in
\epsffile{f1.epsi}}
\end{center}
\caption{Schematic representation of the interaction between WinGraphics and the physics kernel.}
\label{PICFLOW}
\end{figure}
%
\noindent
      The code is separated into a physics application and the windowing core
      as shown in Figure 1.  New physics and diagnostics can be added without
      altering the windowing code, with the only restriction that any new
      diagnostic must be a linear, semi-log, or scatter plot.  A text plot is
      currently under consideration which would display parameters from the
      input file during the simulation.
\vspace{.2in}

\noindent
      Using the windowing core, all diagnostics are updated dynamically in
      time.  The core can also update in individual time-steps, pausing for a
      keystroke before continuing the simulation.
\vspace{.2in}

\noindent
      The code runs significantly faster when fewer (or zero) plots are
      displayed on screen.  Therefore, when output is not being viewed (long
      runs overnight, coffee break, etc.), it is more efficient to remove all
      plots from the screen.
\vspace{.2in}

\noindent
      PDP1 no longer has a time step limitation; it can run indefinitely.  All
      time histories are combed periodically such that there are never more
      than HISTMAX (a programmable constant) values stored.  Note that after
      long runs this can result in a loss of high frequency resolution on the
      time history plots.  This has no effect on the physics of the simulation
      since the diagnostics are simply a view of the physical results.
\vspace{.2in}

\noindent
      PDP1 can now generate hard copy output on an IBM Graphics Printer or
      compatible, including the Epson FX series of dot matrix printers.  PDP1
      can generate a CGM format file for further editing by many drawing
      packages which support ANSI standard CGM and for inclusion in many word
      processors and desktop publishing programs supporting CGM.  In addition
      PDP1 can generate a Postscript file which can be sent to any Postscript
      printer or included in a Postscript document.  The resolution and
      grey-shading of the CGM and Postscript output is dependent only on the
      output device, while the dot matrix output is at the screen resolution in
      monochrome.
\vspace{.2in}

\noindent
      PDP1 now employs four circuit solvers to handle the full range of
      external circuit parameters.  The general voltage-driven series RLC case
      is solved using a second order backward Euler method.  For the open
      circuit case, $C \rightarrow 0$, the external circuit no longer needs to be
      solved; the boundary surface charge influences the potential as always,
      but it cannot exchange charge (via current) with the other boundary.  For
      $C \rightarrow \infty$ and $R = L = 0$, the external circuit becomes a
      short circuit, so the applied potentials are applied directly to the
      plasma boundaries.  The final case is an ideal current source, which
      imposes the specified current independent of the external circuit
      parameters or boundary potentials.  For a complete discussion of the
      simultaneous solution of the external circuit and the spatial plasma
      potential, refer to the attached paper\cite{kn:alves}.
\vspace{.2in}

\noindent
      FFT routines are used to transform time histories into frequency domain
      plots.  Both amplitude and phase of the transformed quantity are
      available.  This feature is useful when looking for response of a bounded
      plasma at one or more frequencies, as well as determining the impedance
      of the plasma to external currents/voltages.
\end{subsubsection}

\begin{subsubsection}
{\bf Version 2.0}

      Version 2.0 supports a number of enhancements, including a Monte Carlo
      collisional model for electron- and ion-neutral collisions.  Dump files
      are now supported.   The current simulation state can be saved to a file
      for later restart.  This is useful for saving long runs toward
      equilibrium, and changing a few parameters and restarting close to the
      new equilibrium.

\vspace{.2in}

\noindent
      In addition, the right boundary of the simulation is now assumed to be
      the zero reference potential in PDP1.  In PDC1 and PDS1, the left
      boundary represents the inner (driven) electrode, while the right
      boundary is the zero reference potential as in PDP1.  A flag has been
      added for refluxing at the right boundary as an alternative to the
      reference potential; refluxing is explained in detail in 4.1.3 Flags.
\end{subsubsection}

\begin{subsubsection}
{\bf Version 2.1}

      An error was discovered in the Monte Carlo collisions (MCC) between
      charged particles and neutrals in calculating the scattered velocity from
      the incident velocity.  In this calculation, one must rotate the velocity
      vector and adjust its magnitude accordingly.  In the rotation part, a set
      of orthonormal vectors is needed to set up a coordinate system to rotate
      the incident velocity vector by theta and phi to the scattered velocity
      vector.  This set of orthonormal vectors was NOT calculated properly
      (they were orthogonal to each other but NOT normal). The vector set that
      we used was 
[
$\vec{v}, (\vec{v} \, {\bf \times} \, \vec{y})$, 
$\vec{v} {\bf \times} (\vec{v} \, {\bf \times} \, \vec{y})$
]  
which should have been 
[
$\vec{v} / |\vec {v}|$, 
$(\vec{v} \, {\bf \times} \, \vec{y}) / |\vec{v} \, {\bf \times} \, \vec{y}|$,
$(\vec{v} \, {\bf \times} \, (\vec{v} \, {\bf \times} \, \vec{y})) / 
|\vec{v} \, {\bf \times} \, (\vec{v} \, {\bf \times} \, \vec{y})|$  
].
	If the vector set is not normalized, the magnitude of the
	scattered velocity
      is also affected through the rotation.  In fact the scattered velocity
      will be smaller than the incident velocity.  This would act as a cooling
      mechanism for the particles, and the system would loose energy.
\vspace{.2in}

\noindent
      In addition a digital smoothing filter (1, 2, 1) is added to the planar
      case (PDP1) which requires an flag in the input file (see the list of
      input parameters).
\end{subsubsection}

\begin{subsubsection}
{\bf Version 3.0 (XPDP1)}

      XPDP1 is the X-Windows version of XPDP1 running under X11.  This version
      makes extensive use of a mouse for selecting menus, moving windows,
      resizing, etc.  In addition, XPDP1 can be run on a remote host while the
      output displays on a local graphics workstation.  XPDP1 uses XGrafix for
      the graphics display, which requires X11 libraries to compile.
\end{subsubsection}

\begin{subsubsection}
{\bf Version 3.1 (XPDP1)}

      In the previous versions, the code would allocate position and velocity
      arrays for each species based on a parameter called MAXLEN (defined
      previously in the header file) which was the maximum number of particles
      for all species.  In that case, the user would have had to recompile the
      whole code if more particles were desired.
\vspace{.2in}

\noindent
      The input files in this version have an extra parameter per species
      (called max-np) for the maximum number of particles of that species.
      This avoids the complications of recompiling and at the same time allows
      the user to define variable array sizes per species.
\end{subsubsection}

\begin{subsubsection}
{\bf Version 4.0 (XPDP1)}

      Xgrafix has been updated to use Tcl/Tk libraries for dialog boxes. 
      The installation of Xgrafix now requires that these libraries be
      also installed.
\vspace{.2in}

\noindent
      The input files in this version has been many new parameters. The 
      major one being {\bf GAS}. This parameter picks out one of the four
      gases built in to PDP1 along with the respective collisions. As a 
      result of this, the collision part of the input file is obsolete
      and no longer required. This subsection is explained in later sections
      for continuity and for help in understanding collision implementation.
      The gases are enumerated below.
\begin {itemize}
\item	GAS = 1: Helium
\item	GAS = 2: Argon
\item	GAS = 3: Neon 
\item	GAS = 4: Oxygen \footnote{Gas = Oxygen requires that {\em nsp} = 3}
\end {itemize}
	In addition to this, an implicit mover {\em imp}, volume source 
{\em psource} and sub cycling {\em k} have also been implemented.

\end{subsubsection}

\begin{subsubsection}
{\bf Version 4.1 (XPDP1)}

\noindent
	A second order accurate injection push has been implemented.  Also all
	components of velocity can be specified independently.  The inversion
	of the distribution functions is now second order accurate.
	The gas package now includes a fifth option (GAS = 5) which allows
	the use of the old cross section parameters (which are explained
	in section 4.1.5-6).  These parameters need to be added to mcc.cpp 
	before compilation.

\end{subsubsection}

\end{subsection}
\end{section}
\newpage
\begin{section}
{\bf INSTALLATION}

This section describes the contents of the \mbox{XPDP1} distribution 
and its installation procedure. 

\begin{subsection}
{\bf Distribution Format}

   XPDP1 is no longer distributed on floppy diskettes. Instead the preferred
distribution procedure is through anonymous FTP. The codes are available
in the tar gzipped form from the {\bf PTSG} web site at 
{\em http://ptsg.eecs.berkeley.edu}. {\bf xgrafix} distribution is available
separately at the same web site and has to be installed for any of the PTSG
family of codes to work. The contents of the XPDP1 distribution 
contains:

\begin{tabbing}
	XPDP1.TARaaaaa  \= Text file containing the direction for installing 
			\mbox{XPDP1} on a Unix \= \kill
        README	\> Text file containing the direction for installing 
			\mbox{XPDP1} on a Unix \> \\
	        \> platform. \> \\
				\\
	XPDP1.TAR  \> This tar file contains the files required 
			for \mbox{XPDP1}. \>
\end{tabbing}

\noindent
\verb1\1{\bf xpdp1} directory:

\begin{tabbing}
        READMEaaaaa\=     All files with the .c extension are the C language
                       source files for \= \kill
        README   \>    Text file containing information about licensing. \> \\
								 	    \\
	*.c      \>    All files with the .c extension are the C language
                       source files for \> \\
		 \>    \mbox{XPDP1}.  These files should be
                       placed in the \mbox{xpdp1} directory. \> \\
								\\
        *.h      \>    All files with the .h extension are the 
			C language header files for \> \\
		 \>    \mbox{XPDP1}.  These files should be
                       placed in the \mbox{xpdp1} directory. \> \\
								\\
        makefile \>    The make file for automatically performing
                       conditional compilation \> \\
		 \>    /linking of only those files
                       which have been changed.  This file \> \\
		 \>    should be placed in the \mbox{xpdp1} directory. \> 
\end{tabbing}

\noindent
\verb1\1{\bf xpdp1}\verb1\1{\bf inp} directory:

\begin{tabbing}
        *.inpaaaaa\=      All files with the inp extension are input files.
                       For detailed information on \= \kill
        *.inp   \>      All files with the inp extension are input files.
                       For detailed information on \> \\
	        \>      each input file refer
                       to Section 4.2 Input File Library.  \= \\
	        \>      A directory called inp is set up
                       under the xpdp1 directory to include all the input
                       files. \> 
\end{tabbing}

\noindent
\verb1\1{\bf xpdp1}\verb1\1{\bf doc} directory:

\begin{tabbing}
        xpdp1.txtaaaaa\=   \mbox{ASCII} file containing the documentation. \= \kill
        xpdp1.ps  \>   \mbox{PostScript} file containing the documentation. \> \\
									\\
        xpdp1.txt \>   \mbox{ASCII} file containing the documentation. \>
\end{tabbing}

\noindent
The {\bf xgrafix} distribution  contains the required files 
for displaying graphics   in the X Windows version.  These are:

\begin{tabbing}
        xgrafix.sd*aaaaa\= The \mbox{XGrafix} icon (bitmap).  This file should be
                       placed in the \mbox{xgrafix} \= \kill

        README     \>  Text file containing some directions and
                       information on compiling \> \\
		   \>  the \mbox{XGrafix} libraries.
                       When the information in this manual conflicts \> \\
		   \>  with
                       the \mbox{README} file, assume the file is correct. \> \\
									\\
        xgrafix.c  \>  The source file for the \mbox{XGrafix} 
				graphics display
                       library.  This file \> \\
		   \>  should be placed in the 
			\mbox{xgrafix} directory. \> \\
							\\
        xgrafix.h  \>  Header file for \mbox{XGrafix}.  This file should be
                       placed in the \mbox{xgrafix} \> \\
		   \>  directory. \> \\
						\\
        xgrafix.ico \> The \mbox{XGrafix} icon (bitmap).  This file should be
                       placed in the \mbox{xgrafix} \> \\
		    \> directory. \> \\
							\\
        xgrafix.str \> Another header file containing string definitions
                       for XGrafix.  This \> \\
		    \> file should be placed in the xgrafix directory. \> \\
								\\
        makefile    \> The make file for \mbox{XGrafix}.  This file should be
                       placed in the \mbox{xgrafix} directory. \> 
\end{tabbing}

\end{subsection}

\begin{subsection}
{\bf Setup and Installation Procedure (X Windows version)}

   The installation procedure to a workstation must be done manually.  Take the
   tar file along with the README file to your workstation or Unix platform.
   Follow the directions in the README file for installation and compiling.
   Read the README file in the xgrafix directory as well as the makefile before
   compiling XGrafix, then compile xpdp1.
\end{subsection}
\end{section}
\newpage
\begin{section}		% begin 3
{\bf X-WINDOWS PROGRAM OPERATION}

The X-Windows version of PDP1, XPDP1, is operated in the same manner as
discussed in the manual for the PC version, with the exceptions noted below.

\begin{subsection}
{\bf Syntax}

   xpdp1 -i filename[.inp] -d [dumpfile.dmp]
 \vspace{.2in}

\noindent
   where $<filename.inp>$ is the name of the input file.  Although we have used
   *.inp for the input files in the library, the .INP extension is not
   required.  If no filename is provided on the command line, XPDP1 displays an
   error message.  The dumpfile parameter is optional; it must be an existing
   file created by the same version of the code.  If the input files are not in
   the same directory or are located in a sub-directory, the path must also be
   specified.  For instance, the syntax for starting XPDP1 with the input file
   vc.inp which is in a sub-directory of xpdp1 called INP is:
\vspace{.2in}

\noindent
   xpdp1 -1 inp/vc[.inp]
\vspace{.2in}

\noindent
   The input file is required since XPDP1 determines the parameters of the
   simulation at run time.
\end{subsection}

\begin{subsection}
{\bf GUI Support}

   XPDP1 fully supports a mouse for selection of items, buttons etc.  Moving,
   resizing, and iconifying of windows is supported indirectly via the X window
   manager (Motif, Open~Look, etc.).  Keystrokes are not supported for these
   actions, so a mouse is required.  The move, resize, and iconifying buttons
   and operations are governed by the window manager; consult the window
   manager manual or guru for details of these procedures.
\end{subsection}

\begin{subsection}
{\bf Main Menu}

   The buttons on the main menu can be selected using the mouse.  The functions
   available include RUN, STOP, STEP, SAVE, and QUIT, which all perform the
   same function described previously in Section 3.  Note that the SAVE
   function is equivalent to the DUMP function in the MS-DOS version which is
   also NOT implemented in this version.
\end{subsection}

\begin{subsection}
{\bf Diagnostic Window Buttons}

Every diagnostic window in XPDP1 contains four buttons: Rescale, 
Trace, Print, and Cross-hair.

\begin{subsubsection}	% begin 3.4.1
{\bf Rescale}

      The rescale button pauses the simulation and opens a dialog box
      containing editable fields for the minimum and maximum labels on the x
      and y axes.  In addition, the dialog box contains buttons for automatic
      rescaling of the x and y axis.  These buttons toggle auto rescaling of the
      respective axis on and off.  When all axes are scaled as desired, select
      OK to accept the changes or CANCEL to return to the previous status.
      Note that while rescaling the simulation is paused.
\end{subsubsection}	% end 3.4.1

\begin{subsubsection}	% begin 3.4.2
{\bf Trace}

      The trace button turns toggles the plot tracing feature on and off.  The
      previous plots are accumulated, generating a series of lines or dots as
      described above.
\end{subsubsection}	% end 3.4.2

\begin{subsubsection}	% begin 3.4.3
{\bf Print}

      The Print button generates a PostScript plot file of the current window.
      Pressing the button opens a dialog box containing the file name for the
      plot and a plot title.  Selecting OK generates the plot, CANCEL returns
      to the simulation.  Note that the simulation is paused while the dialog
      box is open.
\end{subsubsection}	% end 3.4.3

\begin{subsubsection}	% begin 3.4.4
{\bf Crosshair}		

      The crosshair button activates the crosshair pointer and opens a dialog
      box displaying the coordinates of the pointer.  To display the
      coordinates of a point move the crosshair pointer to the desired location
      and click.  The simulation is paused until the crosshair is deactivated
      by selecting the Crosshair button again.
\end{subsubsection}	% end 3.4.4

\end{subsection}	% end 3.4

\begin{subsection}	% begin 3.5
{\bf Diagnostics}

   In XPDP1 diagnostics there is no diagnostic menu list.  Instead, all
   diagnostics appear as icons at the bottom of the display or in the
   designated icon area depending on the window manager.  To open a diagnostic,
   simply click on its icon.  In addition, some window managers will display a
   list of the available diagnostics with all other open windows in a window
   list.  Note that this is not a feature of XPDP1, but rather a feature of the
   X-Windows Manager used on the system.

\end{subsection}	% end 3.5
\end{section}		% end 3

\newpage
\begin{section}		% begin 4
{\bf INPUT FILES}

XPDP1 obtains its versatility through the use of input files.  The input file
contains the parameters for the simulation, specifying number of each species,
grid spacing, charge to mass ratios, etc.  This section describes the contents,
use, and modification of input files for XPDP1.

\begin{subsection}	% begin 4.1
{\bf Input File Parameters}

   The codes use input files to describe the simulation, including the physical
   bounded device parameters, external circuit, RF drive, etc. (global
   parameters), as well as the parameters describing each species of
   particles.  Units, if any, are shown in [ ].

 \begin{subsubsection}
{\bf Global Parameters}
\begin{tabbing}
      epsilonraaaaa\=    The number of particle species to simulate (0= no species
                    present, may \= \kill
     nsp    \>       The number of particle species to simulate (0= no species
                    present, may \> \\
	    \>      use this option to check the system, 1= one
                    species in the whole system, \> \\
	    \>      etc.).  If modifying an input
                    file that has, say, 2 species, to add more \> \\
	    \>      species, just
                    copy one of the blocks of parameters corresponding to
                    species \> \\
	    \> 	    1 or 2, and change the parameters to the desired
                    values.  Note that each \> \\
	    \>      species added seeks another 100
                    kByte of memory. \> \\
					\\
      nc    \>      The number of spatial cells.  The cell width is calculated using \> \\
       	    \>      {\em$\Delta x = length / nc$}. \> \\
							\\
      nc2p  \>      The number of physical particles per computer particle.
                    The number of \> \\
	    \>      super particles in the simulation is found
                    using \> \\
	    \> 	    {\em$N = initn \, \cdot \, area \, \cdot \, length\, /nc2p $}
			, where
                    {\em initn} is the uniform number density. \> \\
							\\
      dt    \>      The time step [sec]. \> \\
						\\
      length \>     The length of the system (distance between electrodes) [m]. \> \\
									\\
      area   \>     Electrode area [{\rm m$^{2}$}]. Allows application of real currents
                    and real external \> \\
	     \>     circuit parameters. \> \\
						\\
      epsilonr \>   Background relative dielectric constant of system. \> \\
							\\
      B   \>        Applied homogeneous magnetic field [Tesla]. \> \\
							\\
      PSI \>        Angle the B-field makes with the normal from the electrode,
                    x-axis \> \\
	  \>        (B-field is in the x-z plane) [deg]. \> \\
							\\
      rhoback \>    Fixed background charge density (non-accelerating) [C/{\rm m$^{3}$}]. \> \\
							\\
      backj   \>    Background current density (non-accelerating) [Amps/{\rm m$^{2}$}]. \> \\

      dde  \>       Sinusoidal perturbation of charge density
                    {\em$(\delta x / l)$} 
			at 
		   	{\em$t = 0$}; \> \\
	   \>       {\em$\delta x (x) = $l$ \, \cdot \, dde \, \cdot \, \sin(2 \pi x/l) $} \> \\
										\\
      extR \>       External circuit resistance [Ohms]. \> \\
								\\
      extL  \>      External circuit inductance [Henries]. \> \\
								\\
      extC  \>      External circuit capacitance [Farads]. \> \\
									\\
      q0   \>       Initial capacitor charge [C]. \> 
\end{tabbing}

\end{subsubsection}

\begin{subsubsection}
{\bf Applied Voltage Or Current Sources}

      When the flag dcramped is off, the general form of the applied source is:

\begin{center}
{\em$S(t) = DC + Ramp \, \cdot \, t + AC \, \cdot \, \sin(2\pi f_{0}t+ \theta_{0})$} 
\end{center}

\noindent
where {\em$S(t)$}, the applied source, is either a current or a voltage source.
\vspace{.2in}

\noindent
      The flag dcramped should be turned on (set to 1) when a step function is
      desired.  The step function can have a zero rise time, $Ramp \gg 1$,
      or can be ramped to its final DC value either with a constant slope or
      sinusoidally as shown in Figure 2. The rise time of the applied signal is
      equal to 1/2f0, in the sinusoidal case, or DC/Ramp, in the constant
      linear slope case.
%
\begin{figure}
\begin{center}
\leavevmode
\hbox{%
\epsfysize 4.0in
\epsfxsize 4.0in
\epsffile{f2.epsi}}
\end{center}
\caption{ When the flag dcramped is set, the signal is ramped to its
final DC value sinusoidally (left), or with a constant slope (right).}
\label{SINRAMP}
\end{figure}
%
\noindent
\begin{tabbing}
      dcrampedaaaaa\=  Flag for ramping external voltage/current source to a final
                    DC value \= \kill
      source    \>  Voltage or current source indicator (v=voltage, i=current). \> \\
						\\
      dcramped  \>  Flag for ramping external voltage/current source to a final
                    DC value \> \\
	        \>  (1=yes, 0=no). \> \\
						\\
      DC        \>  DC voltage or current source [V or Amps].  Zero value
                    indicates zero dc \> \\
		\>  voltage. \> \\
						\\
      Ramp      \>  Rate of ramping for voltage or current source [V/sec or
                    Amps/sec].  Zero \> \\
	        \>  value indicates zero ramping for voltage. \> \\
					\\
      AC        \>  AC voltage or current source [V or Amps]. Zero value
                    indicates zero ac \> \\
		\>  voltage and the values of {\em $f0$} and {\em $theta0$}
                    are ignored. \> \\
					\\
      f0        \>  AC source driving frequency [Hz]. \> \\
					\\
      theta0     \>  Initial phase angle of AC source [deg]. \>
\end{tabbing}
\end{subsubsection}

\begin{subsubsection}
{\bf Flags}

\begin{tabbing}
      e\_collisionalaaaaa\= The flag for ionization, elastic, and excitation
                    electron-neutral collisions \= \kill
      secondary \>  Secondary electron emission flag (0=off, 1=species 1
                    emitted, etc.).  \> \\
		\>    The emitted electron species give the
                    emitted velocity distribution at the \> \\
		\>    surface specified for
                    the species (see SPECIES PARAMETERS). \> \\
							\\
      e\_collisional \> The flag for ionization, elastic, and excitation
                    electron-neutral collisions \> \\
		\>  (0 = off, 1 = species 1 is the
                    colliding electron species, etc.). \> \\
                \>  Note: Only ONE species can be the colliding electron
                    species. \> \\
								\\
      i\_collisional \> The flag for scattering and charge exchange ion-neutral
                    collisions (0 = off, \> \\
		\>  2 = species 2 is the colliding ion
                    species, etc.). \> \\
                \>  Note: Only ONE species can be the colliding ion species. \> \\
						\\
      reflux    \>  The flag for refluxing the particles at the right wall
                    (0=off, 1=on).  In this \> \\
		\>  case, the particles hitting the
                    right wall are not absorbed but reflected back  \> \\
		\>  into the
                    system.  Since the right wall in this case does not charge
                    up, it  \> \\
		\>  serves only as a symmetry plane allowing for a
                    semi-infinite plasma at \> \\
		\>  the right wall.  The particles of
                    each species are refluxed at the temperature \> \\
		\>  specified for
                    the species. \> \\
							\\
      nfft      \>  Number of samples for the Fast Fourier Transform analyzer
                    (must be a \> \\
		\>  power of 2).  When this parameter is set to
                    zero, no FFT analysis is done, \> \\
		\>  and the diagnostics in the
                    frequency-domain are NOT shown. \> \\
									\\
		n\_ave \> Number of samples for the average diagnostics.  When this parameter is set \>\\
		\> to zero, no averages are not done and NOT shown. \> \\
						\\
      nsmoothing \> Number of time that a (1, 2, 1) digital smoothing filter is
                    applied to the \> \\
		\>  charge density arrays prior to the
                    field-solve.  This filter is only present \> \\
		\>  in PDP1. \> \\
\end{tabbing}
\end{subsubsection}

\begin{subsubsection}
{\bf Wall Emission Coefficients and Neutral Gas Parameters}

\begin{tabbing}

      seec(elect.)aaaaa\=  The coefficient of secondary electron emission due to the
                    first species \= \kill

      seec(elect.) \>  The coefficient of secondary electron emission due to the
                    first species \> \\
		   \> striking the two electrodes.  If this
                    parameter is set to say 0.1, on average \> \\
		   \> one electron is
                    injected for every 10 incident particles of this species. \> \\
								\\
      seec(ions)   \>  The coefficient of secondary electron emission due to the
                    second species \> \\	
		   \> striking the two electrodes. \> \\
							\\
      ion species  \>  indicates the ion species created by electron-neutral
                    ionization collisions \> \\
		   \> (2=the created ions are of type
                    species 2, etc.). \> \\
                   \>  Note: this also specifies the type of the background
                    neutral gas particles \> \\
		   \> colliding with electrons. \> \\
						\\
      Gpressure    \>  Background neutral gas pressure [Torr]. \> \\
						\\
      Gtemp        \>  Background neutral gas thermal temperature [eV]. \> 
\end{tabbing}
\end{subsubsection}

\begin{subsubsection} {\bf Electron-Neutral Collisional Parameters}

      The cross-sections \footnote {These details are built-in to XPDP1 in 
      version 4.0} used in the code are close fits to the values measured
      experimentally.  The general expression for all electron-neutral collision
      cross-sections, as seen in Figure 3, is:
\begin{eqnarray*}      
\sigma (E) =  0 & E  <  E_{0}&    \\ 	
\sigma (E)  \propto  E &   E_{0} \leq E  \leq E_{1}&  \\ 	
\sigma (E)  =  \sigma_{max} &  E_{1} \leq E   \leq E_{2}& \\ 	
\sigma (E)  \propto   {\ln(E)  \over E}&  E_{2}\leq E&	     
\end{eqnarray*}

\vspace{.2in}
%
\begin{figure}
\begin{center}
\leavevmode
\hbox{%
\epsfysize 4.0in
\epsfxsize 4.0in
\epsffile{f4.epsi}}
\end{center}
\caption{
General profile for electron-neutral c versus incident particle energy E.
}
\label{Elect}
\end{figure}
%
\begin{tabbing}
   elsengy0aaaaa\=Maximum electron-neutral elastic cross section
		[$m^{2}$]. \= \kill 					
   selsmax\> 	 Maximum electron-neutral elastic cross section
		 [m$^{2}$]. \> \\ 					
				\\ 
   elsengy0\> 	
Elastic collision threshold energy [eV]. \> \\ 			\\
   elsengy1\>	 Low energy of plateau for elastic collisions [eV]. \> \\
				\\ 
   elsengy2\>	 High energy of plateau for elastic collisions [eV]. \> \\ 
					\\ 
   sextmax\>	 Maximum excitation cross section [m$^{2}$]. \> \\
\\ extengy0 \>	Excitation threshold energy [eV]. \> \\
\\ extengy1 \>	Low energy of plateau for excitation [eV]. \> \\
\\ extengy2 \>	High energy of plateau for excitation [eV]. \> \\
\\ sionmax \>	Maximum ionization cross section [m$^{2}$]. \> \\
\\ ionengy0 \>	Ionization threshold energy [eV]. \> \\
					\\
      ionengy1 \> Low energy of plateau for ionization [eV]. \> \\
\\ ionengy2 \> High energy of plateau for ionization [eV]. \>
\end{tabbing}
\end{subsubsection}

\begin{subsubsection} {\bf Ion-Neutral Collisional Parameters} 

      The cross-sections \footnote{These details are built-in to XPDP1 
in version 4.0} for ion-neutral collisions are assumed to be
in the form $\sigma = a + b / \sqrt{E}$, as seen in Figure 4, where a
and b are determined by finding the best fit to the experimental
values.  This empirical fit seems to closely resemble the experimental
cross-sections measured for helium, argon and neon.
%
\begin{figure}
\begin{center}
\leavevmode
\hbox{%
\epsfysize 4.0in
\epsfxsize 4.0in
\epsffile{f3.epsi}}
\end{center}
\caption{
General profile for ion-neutral collision cross-section versus incident particle energy E.
}
\label{IXSECXN}
\end{figure}
%
\noindent
\begin{tabbing}
      achrgxaaaaa\= Charge exchange cross section [m$^{2}$]. \= \kill
      achrgx \> Charge exchange cross section [m$^{2}$]. \> \\
\\ bchrgx \> Charge exchange cross section [m$^{2} \sqrt{e Volt}$].
\> \\ 							\\ ascat \>
Scattering cross section [m$^{2}$]. \> \\
\\ bscat \> Scattering cross section [m$^{2} \sqrt{e Volt}$]. \>
\end{tabbing}
\end{subsubsection}

\begin{subsubsection}
{\bf Species Parameters}

      One set for each species should be specified.

\begin{tabbing}
      max-npaaaaa\= The maximum number of particles per species. \= \kill
      max-np \> The maximum number of particles per species. \> \\
\\ q \> Charge per physical particle [C]. \> \\
\\ m \> Mass per physical particle [kg]. \> \\
\\ j0L \> Magnitude of injected current density from the left
electrode [Amps/m$^{2}$]. \> \\
\\ j0R \> Magnitude of injected current density from the right
electrode [Amps/m$^{2}$]. \> \\
\\ initn \> Initial species physical density in the system [m$^{-3}$].
\> \\
\\ max-np \> The maximum number of particles per species. \>

\end{tabbing}

\noindent
{\bf 4.1.7.1 Velocity Distribution} \\
%
\begin{figure}
\begin{center}
\leavevmode
\hbox{%
\epsfysize 4.0in
\epsfxsize 4.0in
\epsffile{f5.epsi}}
\end{center}
\caption{
Velocity distribution function in x-direction.  The
distribution function in the perpendicular direction does not have a
cutoff, but may have a drift.
}
\label{FOFVX}
\end{figure}
%
\begin{tabbing}
         vyt=vztaaaaa\= Perpendicular thermal velocity Cartesian
components \= \kill

         vx0L \> Drift velocity for v $>$ 0 particles [m/sec]. \> \\
\\ vx0R \> Drift velocity for v $<$ 0 particles [m/sec].  \> \\
\\ vxtL \> Thermal velocity for v $>$ 0 particles [m/sec]. \> \\
\\ vxtR \> Thermal velocity for v $<$ 0 particles [m/sec]. \> \\
\\ vxcL \> Cutoff velocity for v $>$ 0 thermal distribution [m/sec]. \>
\\ 								\\
vxcR \> Cutoff velocity for v $<$ 0 thermal distribution [m/sec]. \> \\
\\ v0y \> Drift velocity in the y directions for particles [m/sec]. \> \\
\\ vty \> Thermal velocity in the y direction for particles [m/sec].  \> \\
\\ v0z \> Drift velocity in the z direction for particles [m/sec]. \> \\
\\ vtz \> Thermal velocity in the z direction for particles [m/sec]. \> \\
\\ 	
\end{tabbing}
Note that the thermal velocity components are defined for a
Maxwellian distribution as 
$v_{thermal} = \left( \frac{k_{B} T}{m} \right)^{1/2}$, 
where $k_{B}$ is Boltzmann's constant, $T$ is the temperature, and
$m$ is the mass of the species.

\noindent
{\bf 4.1.7.2 Energy Distribution Diagnostics} \\

\begin{tabbing}
         Eminaaaaa\= The minimum energy seen in the energy distribution
diagnostic of the \= \kill

        nbin \> Number of bins for the energy distribution diagnostic
of the \> \\ 	 \> species at left wall. \> \\
\\ Emin \> The minimum energy seen in the energy distribution
diagnostic of the \> \\ 	 \> species at left wall [eV]. \> \\
\\ Emax \> The maximum energy seen in the energy distribution
diagnostic of the \> \\ 	 \> species at left wall [eV]. \>
\end{tabbing}

\noindent
Parameters for an energy distribution function in the system.
The parameters XStart and XFinish designate a region (a window) in the
space over which the energy distribution is calculated.

\begin{tabbing}
       XFinishaaaaa\= The right boundary of the region over which the
distribution \= \kill
         nbin \> Number of bins for the energy distribution diagnostic
of the species \> \\ 		\> 	in the system. \> \\
\\	 Emin \> The minimum energy seen in the energy distribution
diagnostic of the \> \\ 	 \>	species in the system [eV]. \>
\\ 				\\ Emax \> The maximum energy seen in
the energy distribution diagnostic of the \> \\
\>	species in the system [eV].	\> \\
\\ XStart \> The left boundary of the region over which the
distribution is calculated. \> \\ 				\\
XFinish \> The right boundary of the region over which the
distribution is calculated. \> \\
\end{tabbing}

\noindent
{\bf 4.1.7.3 Velocity Distribution Diagnostics} \\
\begin{tabbing}
         vyt=vztaaaaa\= Perpendicular thermal velocity Cartesian
components \= \kill

 vx\_lower \> Lower velocity for velocity distribution diagnostics [m/sec]. \> \\
\\ vx\_upper  \> Upper velocity for velocity distribution diagnostics [m/sec].  \> \\
\\ nxbin \> Number of bins used. (if 0 diagnostics is turned off.) \> \\
\\ vy\_lower \> Lower velocity for velocity distribution diagnostics [m/sec]. \> \\
\\ vy\_upper  \> Upper velocity for velocity distribution diagnostics [m/sec].  \> \\
\\ nybin \> Number of bins used. (if 0 diagnostics is turned off.) \> \\
\\ vz\_lower \> Lower velocity for velocity distribution diagnostics [m/sec]. \> \\
\\ vz\_upper  \> Upper velocity for velocity distribution diagnostics [m/sec].  \> \\
\\ nzbin \> Number of bins used. (if 0 diagnostics is turned off.) \> \\
\end{tabbing}

\end{subsubsection}
\end{subsection}

\begin{subsection}
{\bf Input File Library}

   Clearly the number of data sets is virtually unlimited.  PDP1 is
accompanied by a number of prepared simulations discussed below.  The
user is encouraged to make a working copy of the data sets to edit
using any ASCII word processor or editor.  Note that the relative
position of the numeric data is important for useful results, but the
number of spaces separating each number is unimportant as long as the
numbers remain on the same line.  Comments may be added on the lines
containing the descriptive text (so long as it remains on a single
line) and at the end of the input file an unlimited number of text
lines are supported.

\begin{subsubsection}
{\bf PIIIA.INP (PIIIH.INP)}

      This input file simulates plasma immersion ion implantation
(sometimes referred to as plasma source ion implantation) in an argon
(hydrogen) plasma.  A single short step potential is applied to the
target.  The electrons are pushed away from the target before the ions
have responded, generating an ion matrix (uniform) sheath.  On a
longer time scale, the ions are accelerated by the resultant electric
field across the sheath and implanted into the target.  As the ions
are implanted, the ion density in the sheath drops.  This causes the
sheath-plasma edge to recede and uncover more ions to increase the ion
density in the sheath and sustain the potential drop across the
sheath.  The velocity of the moving sheath edge depends upon, among
other factors, the pressure of the background neutral gas.  One can
vary the pressure of the neutral gas and observe the difference in the
profiles of ion energy and angular

      distribution at the target.  Electrons and ions are injected and
absorbed at the right wall.  The rate of injection for electrons is
higher than that for ions by the square root of the mass ratio (mi/me)
to reduce the source sheath produced at the right wall.
\end{subsubsection}

\begin{subsubsection}
{\bf QMACH.INP}

      This input file simulates a Q-machine, in which Maxwellian
distributions of electrons and ions are injected from the right
electrode.  The electrons reach the left electrode before the more
inertial ions, charging the (floating) electrode negatively.  After a
few ion transit times, a sheath drop region near the wall can be seen
in the spatial potential, followed by a flat region of essentially
charge-neutral plasma.  Depending on the difference in the electron
and ion injected current densities, a source sheath may be generated
at the injection electrode.
\end{subsubsection}

\begin{subsubsection}
{\bf RFDA.INP (RFDH.INP)}

      This input file simulates a capacitively coupled RF discharge
started with a few ionized particles in an argon (hydrogen) plasma.
Electrons responding to the RF field applied to the electrodes gain
energy and ionize neutrals, building the plasma initially, until some
equilibrium is reached when the electron-ion pairs generated just
balances the loss to the electrodes.  Note the symmetric (but out of
phase) sheath near the wall in the spatial plot of potential.  The
characteristics of the discharge can be changed by varying the neutral
gas pressure.  Also note that there is no physical difference between
the driven electrode and the reference electrode at zero potential in
PDP1.
\end{subsubsection}

\begin{subsubsection}
{\bf VC.INP}

      This input file simulates a virtual cathode oscillations due to
injection of a cold electron beam between shorted electrodes.  The
oscillations are periodic after an initial transient, and can be seen
in any of the time history plots.
\end{subsubsection}
\end{subsection}

\begin{subsection}
{\bf Format}

   The input file is currently a fixed format ASCII file.  The input
files may be edited using any ASCII text editor or word processor
which does not insert formatting characters into the file.  The text
lines can contain any descriptive comments, etc., and the text may
continue for as many lines as desired.  The lines containing numbers
must also remain on a single line and each number can only be
separated by white space, including any amount of spaces and tabs.
The numbers may be in floating point or exponential format (decimal
point and sign are optional).

   The preferred method of trying new input parameters is to COPY the
original data set to a new file name and edit the working copy.  This
will leaves original data set intact for future use and reference.

\end{subsection}
\end{section}

\newpage
\noindent
\begin{appendix}
\begin{section}
{\bf SYSTEM REQUIREMENTS}

\noindent
The X-Windows version requires X11 libraries (X11R4 or any superset of
X11 such as Motif), a C compiler, and an X display or X-terminal. In
addition xgrafix requires Tcl/Tk also be installed. The exact version of
the (public domain) required graphics libraries can be ascertained by 
reading the README file presented in the distribution.

To edit the input files, the user must have an ASCII text editor or
word processor that can edit and save plain ASCII files without
formatting or control characters.
\end{section}

\newpage
\noindent
\begin{section}
{\bf APPENDIX TECHNICAL SUPPORT}

The Plasma Theory amd Simulation Group maintains mailing lists for discussing
the codes developed by it. In order to subscribe to a mailing list\\
send email to\\
{\em  majordomo@langmuir.eecs.berkeley.edu} \\
with the phrase \\
{\bf subscribe $\langle {\rm {\bf list}} \rangle$}\\
in the body of the message.  This will cause you
to be automatically added to the mailing list.\\
{\bf $\langle {\rm {\bf list}} \rangle$} is the name of the mailing list 
you want to subscribe to and can be one of the following 
\begin {itemize}
\item {\bf oopic-users}: mailing list for oopic and xoopic users.
\item {\bf pdp2-users}: mailing list for xpdp2 users.
\item {\bf pdx1-users}: mailing list for xpdp1 and xpdc1 users.
\item {\bf es1-users}: mailing list for xes1 users.
\end{itemize}

The group also maintains a web site containing home pages of group
members and the anonymous FTP directory for accessing codes and can
be  accessed at\\
 {\em http://ptsg.eecs.berkeley.edu} in the {\em pub/codes} directory\\

For technical support, modifications and custom models, contact
{\bf CPS} (Computer Physics Software) via email at 
{\em cps@langmuir.eecs.berkeley.edu}
\end{section}
\end{appendix}

\newpage
\center\begin{thebibliography}{99}

\bibitem{kn:bird}
C. K. Birdsall and A. B. Langdon, {\em Plasma Physics Via Computer Simulation}, (McGraw-Hill 1985, Adam-Hilger 1991 
which has ES1 disk).

\bibitem{kn:hock}
R. W. Hockney and J. W. Eastwood, {\em Computer Simulation Using Particles}, Adam Hilger (1988).

\bibitem{kn:law}
W. S. Lawson, {\em PDW1 User's Manual}, Electronics Research Laboratory 
Report M84/37 (1984).

\bibitem{kn:verb}
J. P. Verboncoeur and V. Vahedi, {\em WinGraphics: An Optimized Windowing 
Environment for Plasma Simulation}, Proceedings of $13^{th}$ 
Numerical Simulation Conference, Sante Fe, NM (1989).

\bibitem{kn:alves}
John P. Verboncoeur, M. Virginia Alves, V. Vahedi, and
C. K. Birdsall, {\em J. Comp. Phys.} {\bf 104}, 321 (1993).
\end{thebibliography}
\end{document}

